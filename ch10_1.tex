\chapter{Geometry in the Plane} \label{chp 10}
Suppose that we are concerned with the motion of a particle as it moves in a plane. At any time $t$ during the motion, the position of the particle is given by its two coordinates, which depend on time, and may therefore be denoted by $x(t)$ and $y(t)$, respectively. The set of points traced out by the particle as it moves during a given interval of time is a curve. The function which describes the position of the particle is called a parametrization, and a curve described by such a function is said to be parametrized. In the first sections of this chapter we shall develop the mathematical theory of parametrized curves, abstracting from the picture of a physical particle in motion. Later we shall return to this application and define the notions of velocity and acceleration of such particles.

Parametrized curves represent an important generalization of the curves encountered thus far as the graphs of functions. As we shall see, a parametrized curve is not necessarily the graph of an equation $y = f(x)$.

\section{Parametrically Defined Curves.}\label{sec 10.1} 
When we speak of the plane in this book, we assume, unless otherwise stated, that a pair of coordinate axes has been chosen. As a result, we identify the set of points in the plane with the set $R^2$ of all ordered pairs of real numbers. A convenient notation for a function $P$ whose domain is an interval $I$ of real numbers and whose range is a subset of the plane is $P : I \rightarrow R^2$. Every function $P : I \rightarrow R^2$ defines two \textbf{coordinate functions,} the functions which assign to every $t$ in $I$ the two coordinates of the point $P(t)$. If we denote the first coordinate function by $f$, and the second one by $g$, then they are defined by the equation


\begin{equation}
P(t) = (f(t), g(t)), \;\;\;\mbox{for every $t$ in $I$.}  
\label{eq10.1.1}
\end{equation}
Conversely, of course, every ordered pair of real-valued functions $f$ and $g$ with an interval $I$ as common domain defines a function $P : I \rightarrow R^2$ by equation (1).

Since the first and second coordinates of an element of $R^2$ are usually the $x$- and $y$-coordinates, respectively, we may alternatively define a function
%542 GEOMETRY IN THE PLANE [CHAP. 1O 
$P : I \rightarrow R^2$ by a pair of equations 

$$
\left \{ \begin{array}{l}
x = f(t), \\
y = g(t),

\end{array}
\right .
$$
\noindent where $f$ and $g$ are real-valued functions with domain $I$. Then, for every $t$ in $I$, we have $P(t) = (x, y) = (f(t), g(t))$. It is also common practice to denote the coordinate functions themselves by $x$ and $y$. When this is done, we do not hesitate to write the equations $x = x(t)$ and $y = y(t)$, and the function $P : I \rightarrow R^2$ is defined by
$$
P(t) = (x(t), y(t)), \;\;\;\mbox{for every $t$ in $I$.}
$$

A function $P : I \rightarrow R^2$ is said to be \textbf{continuous at} $t_0$ if both coordinate functions are continuous at $t_0$. If the coordinate functions are denoted by $x$ and $y$, then we define
$$
\lim_{t \rightarrow t_0} P(t) = (\lim_{t \rightarrow t_0} x(t), \lim_{t \rightarrow t_0} y(t)).
$$

\noindent As a result, the definition of continuity for $P$ is entirely analogous to that for a real-valued function: $P$ is continuous at $t_0$ if $t_0$ is in the domain of $P$ and if $\lim_{t \rightarrow t_0} P(t) = P(t_0)$. As before, the function $P$ is simply said to be \textbf{continuous} if it is continuous at every number in its domain.

A \textbf{curve} in the plane is by definition a subset of $R^2$ which is the range of some continuous function $P : I \rightarrow R^2$. Every curve is the range of many such functions, and, as a result, it is necessary to choose our terminology carefully. We shall call a continuous function $P : I \rightarrow R^2$, a \textbf{parametrization} of the curve $C$ which is the range of $P$, and we shall say that $C$ is \textbf{parametrically defined} by $P : I \rightarrow R^2$. The points of the curve $C$ obviously consist of the set of all points $P(t)$, for every $t$ in $I$. By a \textbf{parametrized curve} we shall mean the range of a specified continuous function $P : I \rightarrow R^2$. Speaking more casually, we shall refer to the curve defined parametrically by

$$
P(t)= (x(t), y(t)), 
$$
\noindent or, equivalently, to the curve defined parametrically by the equations 
$$
\left \{ \begin{array}{l}
x = x(t), \\
y = y(t),
\end{array}
\right .
$$
\noindent for every $t$ in some interval $I$ which is the common domain of the continuous functions $x$ and $y$. If $t$ is regarded as an independent variable, it is called the \textbf{parameter} of the parametrized curve.

%SEC. 1] PARAMETRICALLY DEFINED CURVES 543
\begin{example} Draw the curve defined parametrically by
$$
P(t) = (t^2, t), \;\;\; -\infty < t < \infty.
$$
\noindent This is, of course, also the curve defined by the equations
$$
\left \{ \begin{array}{l}

x = t^2,\\
y = t, \;\;\; -\infty < t < \infty.
\end{array}
\right .
$$
\noindent It is plotted in Figure 1. Since the set of all points $(x, y)$ which satisfy the above two equations is equal to the set of all points $(x, y)$ such that $x = y^2$, we recognize the curve as a parabola.
\end{example}

%Figure I

\putfig{3.25truein}{scanfig10_1}{}{fig 10.1}

\begin{table}
\centering
\begin{tabular}{r|l}
\hline
t    & (x, y) \\ \hline
0   & (0, 0) \\
1   & (1, 1) \\
-1  & (1, -1) \\
2   & (4, 2) \\
-2  & (4, -2) \\ \hline
\end{tabular}
\caption{}
\label{table 10.1}
\end{table}
\medskip

It is worth noting that every curve which we have previously encountered as the graph of a continuous function $f$ can be defined parametrically. The graph is the set of all points $(x, y)$ such that $x$ is in the domain of $f$ and such that $y = f(x)$. This set is obviously equal to the set of all points $(x, y)$ such that

 
\begin{equation}
\left \{ \begin{array}{l}
x = t,\\
y = f(t), \;\;\;\mbox{and $t$ is in the domain of $f$.}   
\end{array}
\right .
\label{eq10.1.2}
\end{equation}

\noindent Hence the graph of $f$ is defined parametrically by equations (2).

A function $P : I \rightarrow R^2$ is \textbf{differentiable at} $t_0$ if the derivatives of both coordinate functions exist at $t_0$. Moreover, following the usual style, we say that $P$ is a \textbf{differentiable function} if it is differentiable at every number in its domain. This terminology is also applied to parametrized curves. That is, a curve defined parametrically by $P : I \rightarrow R^2$ is said to be differentiable at $t_0$, or simply differentiable, according as $P$ is differentiable at $t_0$, or is a differentiable function.
%544 GEOMETRY IN THE PLANE [CHAP. 1O 
\begin{example} Draw and identify the curve $C$ defined parametrically by

$$
P(t) = (x(t), y(t)) = (4\cos t, 3\sin t), 
$$
\noindent for every real number $t$. If $(x, y)$ is an arbitrary point on the curve, then 
 
$$
 \{ \begin{array}{l}
x = 4 \cos t, \\
y = 3 \sin t,
\end{array}
$$

\noindent for some value of $t$. Hence, $\frac{x}{4} = \cos t$ and $\frac{y}{3} = \sin t$, and, consequently, 
$$
\frac{x^2}{16} + \frac{y^2}{9} = \cos^2 t + \sin^2 t = 1 .
$$
\noindent Thus for every point $(x, y)$ on the curve, we have shown that 

\begin{equation}
\frac{x^2}{16} + \frac{y^2}{9} = 1.  
\label{eq10.1.3}
\end{equation}


\noindent The latter is an equation of the ellipse shown in Figure 2, and it follows that the curve $C$ is a subset of the ellipse. Conversely, let $(x, y)$ be an arbitrary point on the ellipse. Then $|x| \leq 4$, and so there exists a number $t$ such that
$x = 4 \cos t$. Since $\cos t = \cos(-t)$ and $\sin t = -\sin(-t)$, we may choose $t$ so that $\sin t$ and $y$ have the same sign. Then, solving equation (3) for $y$ and setting $x = 4 \cos t$, we obtain

%Figure 2
\putfig{3.5truein}{scanfig10_2}{}{fig 10.2}


\begin{eqnarray*}
y^2 &=& 9 \Big(1 - \frac{x^2}{16} \Big) = 9 \Big(1 - \frac{16 \cos^2 t}{16} \Big) = 9(1 - \cos^2 t) \\
      &=& 9 \sin^2t.
\end{eqnarray*}

%SEC, 1] PARAMETRICALLY DEFINED CURVES  545
\noindent Since $y$ and $\sin t$ have the same sign, it follows that $y = 3 \sin t$. We have therefore proved that, if $(x, y)$ is an arbitrary point on the ellipse, then there exists a real number $t$ such that
$$
(x, y) = (4 \cos t, 3 \sin t) = P(t).
$$
\noindent That is, every point on the ellipse also lies on $C$. We have already shown that the converse is true, and we therefore conclude that the parametrized curve $C$ is equal to the ellipse.
\end{example}

Consider a curve $C$ defined parametrically by a differentiable function $P : I \rightarrow R^2$, and let $t_0$ be an interior point of the interval $I$. A typical example is shown in Figure 3. Generally it will not be the case that the whole curve is a function of $x$, since there may be distinct points on $C$ with the same $x$-coordinate. However, it can happen that a subset of $C$ containing the point $P(t_0)$ is a differentiable function. Such a subset is shown in Figure 3, drawn with a heavy line. Thus if $P(t) = (x(t), y(t))$ for every $t$ in $I$, then there may exist a differentiable function $f$ such that

%Figure 3
\putfig{4.5truein}{scanfig10_3}{}{fig 10.3}

\begin{equation}
y(t) = f(x(t)),  
\label{eq10.1.4}
\end{equation}


\noindent for every $t$ in some subinterval of $I$ containing $t_0$ in its interior. If such a function does exist, we shall say that $y$ \textit{is a differentiable function of $x$ on the parametrized curve $P(t) = (x(t), y(t))$ in a neighborhood of the point $P(t_0)$.}  Applying the Chain Rule to equation (4), we obtain
$$
y'(t) = f'(x(t)) x'(t) .
$$
%546 GEOMETRY IN THE PLANE [CHAP. 1O 
\noindent Hence

\begin{equation}
f'(x(t)) = \frac{y'(t)}{x'(t)},
\label{eq10.1.5}
\end{equation}

\noindent for every $t$ in the subinterval, for which $x'(t) \neq 0$. If we write $y = f(x)$ and use the differential notation for the derivative, formula (5) becomes


\begin{equation}
\frac{dy}{dx} = \frac{\frac{dy}{dt}}{\frac{dx}{dt}}. 
\label{eq10.1.6}
\end{equation}
\noindent It should be apparent that $f'(x(t))$, or, equivalently, $\frac{dy}{dx}$ at $t$, is equal to the slope of the curve $C$ at the point $P(t)$.  


\begin{example} Find the slope, when $t = \frac{\pi}{3}$, of the parametrized ellipse in Example 2. The parametrization is defined by the equations
$$
\left \{ \begin{array}{l}
x= 4\cos t, \\
y = 3 \sin t.
\end{array}
\right.
$$
 
\noindent We shall assume the analytic result that $y$ is defined as a differentiable function of $x$ in a neighborhood of the point
$$
\Big(4 \cos \frac{\pi}{3},  3 \sin \frac{\pi}{3} \Big).
$$

\noindent Since

$$
\Big(4 \cos \frac{\pi}{3}, 3 \sin \frac{\pi}{3} \Big) = \Big(4 \cdot \frac{1}{2}, 3 \cdot \frac{\sqrt 3}{2} \Big) = \Big(2 , \frac{3 \sqrt 3}{2} \Big) ,
$$

\noindent one can see by simply looking at Figure 2 that this should certainly be true since the curve passes smoothly through the point and, in the immediate vicinity of the point, does not double back on itself. We have

\begin{eqnarray*}
\frac{dx}{dt} &=& \frac{d}{dt} 4 \cos t = - 4 \sin t ,\\
\frac{dy}{dt} &=& \frac{d}{dt} 3 \sin t = 3 \cos t ,
\end{eqnarray*}

%SEC. 1] PARAMETRICALLY DEFINED CURVES 547

\noindent and so
\begin{eqnarray*}
\frac{dx}{dt}\Big|_{t=\pi/3} &=& -4 \sin \frac{\pi}{3} = -4 \frac{\sqrt 3}{2} = - 2\sqrt 3, \\
\frac{dy}{dt}\Big|_{t=\pi/3} &=& 3 \cos \frac{\pi}{3} = \frac{3}{2} . 
\end{eqnarray*}

\noindent Hence, by formula (6), the slope is equal to
$$
\frac{dy}{dx}\Big|_{t=\pi/3} = \frac{\frac{dy}{dt}\Big|_{t=\pi/3}}{\frac{dx}{dt}\Big|_{t=\pi/3}} = \frac{\frac{3}{2}}{-2 \sqrt 3} = - \frac{3}{4 \sqrt 3} .
$$
\end{example}

The problem of giving analytic conditions which imply that $y$ is a differentiable function of $x$ on a parametrized curve in the neighborhood of a point is akin to the problem of determining when an equation $F(x, y) = c$ implicitly defines $y$ as a differentiable function of $x$ in a neighborhood of a point. As mentioned on page 81, the latter is solved by the Implicit Function Theorem, and the techniques needed here are similar.

As a final example, let us consider the curve traced by a point fixed on the circumference of a wheel as the wheel rolls along a straight line. We take the $x$-axis for the straight line. The radius of the wheel we denote by $a$, and the point on the circumference by $(x,y)$. If we assume that the point passes through the origin as the wheel rolls by to the right, then the curve is
defined parametrically by the equations
$$
\left \{ \begin{array}{l}
x = a(\theta - \sin \theta),\\
y = a(1 - \cos \theta), \;\;\; -\infty < \theta < \infty,
\end{array}
\right .
$$

\noindent where the parameter $\theta$ is the radian measure of the angle with vertex the center of the wheel, initial side the half-line pointing vertically downward, and terminal side the half-line through $(x, y)$ (see Figure 4). (An alternative
geometric interpretation of the parameter is that $a\theta$ is the coordinate of the point of tangency of the wheel on the $x$-axis.) The curve is called a \textbf{cycloid.} Note that the parametric equations are quite simple, whereas it would be difficult to express $y$ as a function of $x$.


%Figure 4
\putfig{4.5truein}{scanfig10_4}{}{fig 10.4}

%548 GEOMETRY IN THE PLANE [CHAP. 1O
