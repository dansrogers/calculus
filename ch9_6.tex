\section{Power Series.}
 Associated with every infinite sequence of real numbers $a_0, a_1, a_2, . . .$ and every real number $x$, there is the infinite series

$$
\sum_{i=0}^\infty a_{i} x^i = a_0 + a_{1} x + a_{2} x^2 + \cdots . 
$$
\noindent Such a series is called a \textbf{power series in $x$}. As a general rule, it will converge for some values of $x$, but not all. For example, the geometric series
$$
1 + x + x^2 + x^3 + . . .
$$
\noindent converges and has the same value as $\frac{1}{1-x}$ for every real number $x$ in absolute value less than 1, but it diverges for all other values of $x$. A power series which converges for some real number $c$, i.e., which converges if $x = c$, is commonly said to converge at $c$. Note that every power series in $x$ converges at 0, since, if $x = 0$, then
$$
\sum_{i=0}^\infty a_{i}x^i = a_0 + a_{1}0 + a_{2}0^2 + \cdots = a_0. 
$$

The following proposition is the basic theorem in studying the convergence of power series:

%(6.1) 
\begin{theorem}  
If a power series $\sum_{i=0}^\infty a_{i}x^i$ converges for some real number $c$, then it is absolutely conuergent for every real number $x$ such that $|x| < |c|$.
\end{theorem}
%512 INFINITE SERIES [CHAP. 9


\begin{proof}
If $c = 0$, the result is vacuously true, so we shall assume that $c \neq 0$. The fact that the series $\sum_{i=0}^\infty a_{i}c^i$ converges implies that $\lim_{n \rightarrow \infty} a_{n}c^n= 0$. Hence there exists a nonnegative integer $N$ such that $|a_{n}c^n| \leq 1$, for every integer $n \geq N$. Since $|a_{n}c^n| = |a_n| |c^n|$, it follows that
$$
|a_n| \leq \frac{1}{|c|^n}, 
$$
and thence that
$$
|a_{n}x^n| = |a_n| |x|^n \leq \frac{|x|^n}{|c|^n}  ,
$$
for every real number $x$ and for every integer $n \geq N$. We now impose the restriction that $|x| < |c|$, and set $r = \frac{|x|}{|c|}$. Then $r < 1$, and  
$$
|a_{n}x^n| \leq r^n, \;\;\;\mbox{for every integer}\; n \geq N.
$$
That is, we have shown that $|a_{n}x^n| \leq r^n$ eventually. Since the geometric series $\sum_{i=0}^\infty r^i$ converges if $|r| < 1$, it follows by the Comparison Test that $\sum_{i=0}^\infty |a_{i} x^i|$ converges. This completes the proof.
\end{proof}

We shall derive three corollaries of (6.1). The first asserts that the set of all real n''mbers $x$ at which a power series $\sum_{i=0}^\infty a_{i}x^i$ \textit{converges is a nonempty interval on the real line.} The set is nonempty because, as is remarked above, it contains the number 0. A set of real numbers is an interval if, whenever it contains two numbers, it contains every number in between those two. Thus we must prove that if the series converges at $a$ and at $c$ and if $a < b < c$, then it also converges at $b$. This is quickly done. Suppose first that $b \geq 0$. Then
$$
|b| = b < c = |c|,
$$
\noindent and (6.1) implies that the series converges at $b$. On the other hand, if $b < 0$, then
$$
|b| = -b < -a = |a|,
$$
\noindent and it again follows from (6.1) that the'series converges at $b$. This completes the proof, and, as a result, we call the set of all numbers at which a power series converges the \textbf{interval of convergence} of the power series.

A number $a$ is called an \textbf{interior point} of a set $S$ of real numbers if there exists an open interval which contains $a$ and which is a subset of $S$. The set of all interior points of $S$ is called \textbf{the interior of} $S$. For example, if $S$ is itself an open interval, then all its points are interior points and hence $S$ equals its own interior. More generally, the interior of an arbitrary interval consists of the interval with its endpoints deleted.

The second corollary states that \textit{a power series converges absolutely at every interior point of its interval of convergence.} The proof is virtually the same as that of the first corollary. Let $b$ be an arbitrary interior point of the interval of convergence. Because it is an interior point, we know there exist
%SEC. 6] POWER SERIES  513
real numbers $a$ and $c$ which also lie in the interval and for which $a < b < c$. As before, if $b \geq 0$, then $|b| < |c|$, but if $b < 0$, then $|b| < |a|$. In either case it follows from (6.1) that the series converges absolutely at $b$, and this completes the argument.

The third corollary is the following: \textit{The interior of the interval $I$ of convergence of a power series $\sum_{i=0}^\infty a_{i}x^i$ is symmetric about the origin.} That is, if $b$ is an interior point of $I$, then so is $-b$. Again, there exist numbers $a$ and $c$ in $I$ such that $a < b < c$. Now consider the open interval $(-c, -a)$. It certainly contains $-b$, and, if we can show that $(-c, -a)$ is a subset of $I$, then we shall have proved that $-b$ is an interior point of $I$. Let $x$ be an arbitrary number in $(-c, -a)$, that is, $-c < x < -a$. There are the, by now familiar, two possibilities: If $x \geq 0$, then $-a > 0$ and
$$
|x| = x < -a = |a|.
$$
\noindent If $x < 0$, then $-c < 0$, whence $c > 0$, and
$$
|x| = -x < c = |c|.
$$
\noindent For either possibility, the convergence of the series at $x$ is implied by (6.1), and so the third corollary is proved.

If a power series converges for every real number, then its interval of convergence is the set of all real numbers. The only other possibility, according to the third corollary above, is that the interval of convergence is bounded with symmetrically located endpoints $-\rho$ and $\rho$. We define the \textbf{radius of convergence} of a power series $\sum_{i=0}^\infty a_{i}x^i$ to be infinite if the interval of convergence is the set $(-\infty, \infty)$ of all real numbers, and to be the right endpoint of the interval of convergence if the interval is bounded. The preceding results are then summarized in Figure 6 below and in the following theorem:

%Figure 6
\putfig{4.5truein}{scanfig9_6}{}{fig 9.6}

%(6.2) 
\begin{theorem} lf a power series $\sum_{i=0}^\infty a_{i}x^i$ has radius of convergence $\rho$, then the series converges absolutely at every $x$ in the open interval $(-\rho, \rho)$. lf $\rho$ is not infinite, then $-\rho$ and $\rho$ are the endpoints of the interval of convergence.
\end{theorem}

It is important to realize that, when $\rho$ is finite, we have made no prediction as to whether the series converges or diverges at $\rho$ and at $-\rho$. All we
%514 INFINITE SERIES [CHAP. 9
know is that it converges absolutely in the open interval $(-\rho, \rho)$ and diverges outside the closed interval $[-\rho, \rho]$.

%EXAMPLE 1. 

\begin{example}

Find the interval and radius of convergence of each of the following power series:
 
\begin{quote}
\begin{description}
\item[(a)] $\sum_{i=0}^\infty \frac{x^i}{i!} = 1 + x + \frac{x^2}{2!} + \frac{x^3}{3!} + \cdots.$
\item[(b)] $\sum_{k=1}^\infty(-1)^{k-1} \frac{x^k}{k} = x - \frac{x^2}{2} + \frac{x^3}{3} - \cdots.$
\item[(c)] $\sum_{n=0}^\infty n!x^n = 1 + x + 2x^2 + 3!x^3 + \cdots.$
\end{description}
\end{quote}
 
\noindent In many examples the radius of convergence can be found easily using the Ratio Test.

For the series in (a), we set $u_i = \frac{x^i}{i!}$ and form the ratio
$$
\frac{|u_{n+1}|}{|u_n|} = | \frac{x^{n+1}}{(n + 1)!} | | \frac{n!}{x^n} | = |x| \frac{n!}{(n + 1)!}.
$$
\noindent Since $(n+ 1)! = (n + 1)n!$,
$$
\frac{|u_{n+1}|}{|u_n|}  = |x| \frac{1}{n + 1}.
$$
\noindent Hence
$$
\lim_{n \rightarrow \infty} \frac{|u_{n+1}|}{|u_n|} = \lim_{n \rightarrow \infty} \frac{|x|}{n + 1} = 0. 
$$
\noindent It follows from the Ratio Test that the series $\sum_{i=0}^\infty \frac{x^i}{i!}$ is absolutely convergent for every real number $x$. Hence the interval of convergence is the entire real line, and the radius of convergence is infinite.

Let $u_k = (-1)^{k-1} \frac{x^k}{k}$ for the series in (b). For every integer $k \geq 1$, we have $|u_k| = \frac{|x|^k}{k}$. Hence
$$
\frac{|u_{k+1}|}{|u_k|} = \frac{|x|^{k+1}}{k+1} \frac{k}{|x|^k} = |x| \frac{k}{k+1} . 
$$
\noindent Since 
$$
\lim_{k \rightarrow \infty} \frac{k}{k+1} = \lim_{k \rightarrow \infty} \frac{1}{1 + \frac{1}{k} } = 1,
$$
%SEC. 6] POWi R SERlkS  515
\noindent we obtain
$$
\lim_{k \rightarrow \infty} \frac{|u_{k+1}|}{|u_k|} = |x| \lim_{k \rightarrow \infty} \frac{k}{k + 1} = |x| .
$$
\noindent The Ratio Test therefore implies that the series $\sum_{k=0}^\infty (-1)^{k-1} \frac{x^k}{k}$ converges absolutely if $|x| < 1$ and diverges if $|x| > 1$. It follows that the endpoints of the interval of convergence are the numbers $-1$ and 1 and that the radius of convergence is 1. If $x = 1$, the series becomes
$$
\sum_{k=1}^\infty (-1)^{k-1} \frac{1}{k} = 1- \frac{1}{2} + \frac{1}{3} - \frac{1}{4} + \cdots .
$$
\noindent which is the convergent alternating harmonic series. On the other hand, if $x = -1$, the series becomes
$$
\sum_{k=1}^\infty (-1)^{k-1} \frac{(-1)^k}{k} = \sum_{k=1}^\infty (-1)^{2k-1} \frac{1}{k} . 
$$
\noindent Since $2k - 1$ is always an odd integer, we have $(-1)^{2k-1} = -1$, and so  
$$
\sum_{k=1}^\infty (-1)^{2k-1} \frac{1}{k} = -\sum_{k=1}^\infty \frac{1}{k},
$$
\noindent which diverges. Hence the interval of convergence of the power series $\sum_{k=1}^\infty (-1)^{k-1} \frac{x^k}{k}$ is the half-open interval $(-1, 1]$.

For the series in (c), let $u_n = n! x^n$. We then get
$$
\frac{|u_{n+1}|}{|u_n|} =  \frac{|(n + 1)! x^{n + 1}|} {| n! x^n |} = (n + 1) |x| .
$$

\noindent It follows that
$$
\lim_{n \rightarrow \infty} \frac{|u_{n+1}|}{|u_n|} = \lim_{n \rightarrow \infty} (n + 1) |x| = \left \{ \begin{array}{ll}
\infty,  & \mbox{if}\; x \neq 0,\\
0,        & \mbox{if}\; x = 0.
\end{array}
\right .
$$
\noindent From the Ratio Test we conclude that the series $\sum_{n=0}^\infty n! x^n$ converges only at $x = 0$. The radius of convergence is therefore equal to 0, and the interval of convergence contains the one number 0.
\end{example}

A significant generalization of the definition of power series can be made as follows: Consider an arbitrary real number a and an infinite sequence of
%516 INFINITE SERIES [CHAP. 9
real numbers $a_0, a_1, a_2, ...$. For every real number $x$, a \textbf{power series in} $x-a$ is defined by

\begin{equation}
\sum_{i=0}^\infty a_i(x - a)^i = a_0 + a_1(x - a) + a_2(x - a)^2 + \cdots
\label{eq9.6.1}
\end{equation}
\noindent The power series in $x$ studied earlier in this section are simply instances of the present definition for which $a = 0$.

Fortunately, it is not necessary to start from the beginning again to develop the theory of convergence of power series in $x - a$. Consider the power series in $y$ obtained by making the substitution $x - a = y$ in the series (1). We obtain

\begin{equation}
\sum_{i=0}^\infty a_{i} y^i = a_0 + a_{1} y + a_{2} y^2 + \cdots . 
\label{eq9.6.2}
\end{equation}
\noindent Let $I$ be the set of all real numbers $y$ for which (2) converges, i.e., the interval of convergence of the power series (2). Similarly, let $J$ be the set of all real numbers $x$ for which (1) converges. Since $x - a = y$, or equivalently, $x = y + a$, a number $b$ will belong to $I$ if and only if $b + a$ belongs to $J$. Thus the set $J$ consists of all numbers of the form $b + a$, where $b$ belongs to $I$. Symbolically, we write
$$
J = I + a.
$$
 
%Figure 7
\putfig{4.5truein}{scanfig9_7}{}{fig 9.7}

\noindent Geometrically, therefore, the set $J$ is obtained by translating the interval $I$ along the real line a distance $|a|$, translating to the right if $a > 0$, and to the left if $a < 0$ (see Figure 7). Hence, the set $J$ of all real numbers $x$ for which $\sum_{i=0}^\infty a_{i} (x - a)^i$ converges is also an interval, and is called the \textbf{interval of convergence} of that series. The corresponding radius of convergence is equal to the \textbf{radius of convergence} of the series (2), but this time it should be interpreted (if it is not infinite) as the distance from the number $a$ to the right
%SEC. 6] POWER SERIES  517
endpoint of the interval of convergence $J$. The interior of $J$ is, of course, symmetric about the number $a$.

A power series $\sum_{i=0}^\infty a_{i}(x - a)^i$ is frequently called a power series about the number $a$. If we recall that $|x - a|$ is geometrically equal to the distance between $x$ and $a$, we see that the appropriate analogue of theorem (6.2) is:

%(6.3) 
\begin{theorem} THEOREM. If the power series $\sum_{i=0}^\infty a_{i} (x - a)^i$ has radius of convergence $\rho$, then the series converges absolutely at every $x$ such that $|x - a| < \rho$. If $\rho$ is not infinite, the series diverges at every $x$ such that $|x - a| > \rho$.
\end{theorem}

%EXAMPLE 2. 
\begin{example} Find the radius and interval of convergence of the power series
$$
\sum_{i=0}^\infty \frac{1}{3^i} (x + 2)^i = 1 + \frac{x + 2}{3} + \frac{(x + 2)^2}{3^2} + \cdots .
$$
\noindent Observe, first of all, that this is a power series about the number $-2$. That is, to put it in the standard form (1), we must write the series as
$$
\sum_{i=0}^\infty \frac{1}{3^i} (x - (-2))^i .
$$
\noindent Let us set $u_n = \frac{1}{3^n} (x + 2)^n$ and apply the Ratio Test. For every integer $n \geq 0$, we obtain
$$
\frac{|u_{n+1}|}{|u_n|} = |\frac{(x + 2)^{n+1}}{3^{n+1}}| |\frac{3^n}{(x + 2)^n}| = \frac{|x + 2|}{3} .
$$
\noindent Hence
$$
q = \lim_{n \rightarrow \infty} \frac{|u_{n+1}|}{|u_n|} = \frac{|x + 2|}{3} .
$$
\noindent The series is therefore absolutely convergent if $q < 1$, or, equivalently, if $|x + 2| < 3$, and it is divergent if $q > 1$, or, equivalently, if $|x + 2| > 3$. Remember that $|x + 2|$ is the distance between $x$ and $-2$. Thus the interior of the interval of convergence is the set of all real numbers
whose distance from $-2$ is less than 3. Hence the radius of convergence is equal to 3, and the endpoints of the interval of convergence are the numbers $-5$ and 1. If $x = - 5$, the series becomes
 \begin{eqnarray*}
\sum_{i=0}^\infty \frac{1}{3^i} (-5+2)^i 
             &=& \sum_{i=0}^\infty \frac{(-3)^i}{3^i} \\
             &=& \sum_{i=0}^\infty(-1)^i = 1 - 1 + 1 - 1 + 1 - \cdots ,
\end{eqnarray*}
%518 INFINITE SERIES [CHAP. 9 
\noindent which diverges. If $x = 1$, we also obtain a divergent series,
$$
\sum_{i=0}^\infty \frac{1}{3^i} (1 + 2)^i =  \sum_{i=0}^\infty 1 = 1 + 1 + 1 + 1 \cdots .
$$
\noindent It follows that the interval of convergence of the power series
$$
\sum_{i=0}^\infty \frac{1}{3^i}(x+2)^i
$$
\noindent is the open interval $(-5, 1)$ (see Figure 8).
\end{example}

%Figure 8
\putfig{4.5truein}{scanfig9_8}{}{fig 9.8}

