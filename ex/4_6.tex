\begin{exercises}

\ex{4.6.1}
Evaluate the following indefinite integrals.
\begin{exenum}
\x
$\int (x^2 + x + 1) \; dx$
\x
$\int \left(3x^2 - \frac1{3x^3}\right) \; dx$
\x
$\int (6t^2 - 2t + 5) \; dt$
\x
$\int (2y + 1)(y - 3) \; dy$
\x
$\int (2x-1)^\frac32 \; dx$
\x
$\int (3x^3 + 2)^5x^2 \; dx$
\x
$\int x\sqrt{a^2 - x^2} \; dx$
\x
$\int \frac{t + 2}{\sqrt{t^2 + 4t + 5}} \; dt$
\x
$\int s(s^3 + 3s^2 + 5)(s + 2) \; ds$
\x
$\int |x| \; dx$.
\end{exenum}

\ex{4.6.2}
Among the following integrals identify
those that can be evaluated using the techniques
in this section.  Evaluate them.
\begin{exenum}
\x
$\int \left(x + \frac1x\right) \; dx$
\x
$\int \left(\sqrt{x} + \frac1{\sqrt{x}}\right) \; dx$
\x
$\int y^2(y^3 + 7)^4 \; dy$
\x
$\int y(y^3 + 7)^4 \; dy$
\x
$\int t\sqrt{t^3 - 1} \; dt$
\x
$\int \frac{x+1}{x-1} \; dx$
\x
$\int (3x^2 - 1)(x + 2) \; dx$
\x
$\int (s+1)(s^2 + 2s - 3)^4 \; ds$
\x
$\int \frac{x^2 - 1}{x + 1} \; dx$
\x
$\int \frac{y + 2}{y^2 + 1} \; dy$
\x
$\int \frac{x - 1}{(x + 1)^3} \; dx$.
\end{exenum}

\ex{4.6.3}
The curve defined by $y = f(x)$
passes through the point $(1,4)$.
In addition, at each point $(x,f(x))$,
the slope of the curve is
$8x^3 + 2x$.  Find $f(x)$.

\ex{4.6.4}
The line tangent to the graph of the differentiable
function $f$ at each point $(x,f(x))$ has slope
$3x^2 + 1$, and the graph passes through the point
$(2,9)$.  Find $f(x)$.

\ex{4.6.5}
If $f^{\prime\prime}(x) = 12x^2 + 2$
and the graph of $y = f(x)$ passes through
$(0,-2)$ with a slope of $5$, find $f(x)$.

\ex{4.6.6}
Evaluate the following definite integrals.
\begin{exenum}
\x
$\int_0^1 (3x^2 + 4x + 1) \; dx$
\x
$\int_{-1}^1 (2t^3 + t) \; dt$
\x
$\int_{-1}^1 (x^3 + 1)^{17}x^2 \; dx$
\x
$\int_{-1}^2 \frac{s+1}{\sqrt{s^2 + 2s + 3}} \; ds$
\x
$\int_1^3 \left(x^2 + \frac1x\right)^3 \left(2x - \frac1{x^2}
\right) \; dx$
\x
$\int_0^2 \frac1{(x+1)^2} \; dx$
\x
$\int_{-2}^2 \sqrt{4-x^2} \, x \; dx$
\x
$\int_{-2}^2 (2|x| + 1) \; dx$
\x
$\int_0^1 t(t^3 + 3t^2 - 1)^3 (t + 2) \; dt$
\x
$\int_1^2 \frac{x^4 + 2x^3 - 2}{x^2} \; dx$.
\end{exenum}

\ex{4.6.7}
If $f^{\prime\prime}(x) = 18x + 10$ and
$f^\prime(0) = 2$, find $f^\prime(x)$.
If, in addition, $f(0) = 1$, find $f(x)$.

\ex{4.6.8}
\begin{exenum}
\x
If $g^{\prime\prime}(x) = \sqrt x$ and
$g^\prime (1) = 0$ and $g(0) = \sqrt2$,
find $g(x)$.
\x
If $f^{\prime\prime\prime}(t) = 6$ and
$f^{\prime\prime}(1) = 8$ and $f^\prime(0) = 1$
and $f(1) = 4$, find $f(t)$.
\end{exenum}

\ex{4.6.9}
If the slope of the curve $y = f(x)$
is equal to $6$ at the point $(1,4)$ and,
more generally, equals $6x$ at $(x,f(x))$,
what is the area bounded by the curve $y = f(x)$,
the $x$-axis, and the lines $x = 1$ and $x = 3$?

\ex{4.6.10}
Sketch the region bounded by the curves
$y = \frac1{\sqrt{x + 1}}$, $x = 0$, and $y = \frac12$.
Find its area.

\end{exercises}
