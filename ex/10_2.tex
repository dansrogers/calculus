\begin{exercises}

\ex{10.2.1}
Find the arc lengths of the following parametrized
curves.
\begin{exenum}
\x
$\dilemma{x = t+1,}
{y = t^{\frac32}, & \mbox{from $(2,1)$ to $(5,8)$.}}$
\x
$\dilemma{x = t^2,}
{y = \frac23 (2t+1)^\frac32, & 
\mbox{from $\left(x(0),y(0)\right) = (0, \frac23)$ to
to $\left(x(4), y(4)\right) = (16,18)$.}}$
\x
$P(t) = (t^2, t^3)$, \quad from $P(0)$ to $P(2)$.
\x
$\dilemma{x(\theta) = a \cos^3\theta, & a>0,}
{y(\theta) = a \sin^3\theta, & 
\mbox{from $\left(x(0), y(0)\right) = (a,0)$ to
$\left(x(\frac{\pi}2), y(\frac{\pi}2)\right) = (0,a)$.}}$
\end{exenum}

\ex{10.2.2}
A particle in motion in the plane has position equal to
\[
P(t) = \left(t^2+t, \frac16(4t+3)^\frac32\right)
\]
at time $t$.  How far does the particle travel
along its path from time $t=0$ to time $t=1$?

\ex{10.2.3}
Find the arc lengths of the graphs of
each of the following functions $f$ between
the points $(a, f(a))$ and $(b,f(b))$.
\begin{exenum}
\x
$f(x) = x^\frac32$, $a=1$, and $b=4$.
\x
$f(x) = \frac23 (x^2+1)^\frac32$, $a=0$, and $b=2$.
\x
$f(x) = x^2$, $a=0$, and $b=\frac12$.
\x
$f(x) = \frac12(e^x + e^{-x})$, $a=-1$ and $b=1$.
\end{exenum}

\ex{10.2.4}
Show that the circumference of an ellipse with the
line segment joining $(-a,0)$ and $(a,0)$
as major axis and the line segment joining
$(0,-b)$ and $(0,b)$ as minor axis is
given by an integral
\[
K \int_0^{2\pi} \sqrt{1+k\sin^2\theta} \; d\theta
.
\]
Evaluate the constants $K$ and $k$ in terms
of $a$ and $b$.
(Do not attempt to evaluate the integral.)

\ex{10.2.5}
\begin{exenum}
\x
Let $g$ be a function which is continuously
differentiable on the closed interval $[c,d]$.
Prove, as a corollary of Theorem \thref{10.2.2},
that the arc length ${L_c}^d$ of the graph
of the equation $x = g(y)$ between the points
$(g(c), c)$ and $(g(d), d)$ is given by the formula
\[
{L_c}^d = \int_c^d \sqrt{1+g^\prime(y)^2} \; dy
.
\]
\x
Find the arc length of the graph of the equation
$x = \frac13(y^2+2)^\frac32$
between the point
$\left( \frac{2\sqrt2}2, 0 \right)$
and the point
$(2\sqrt6, 2)$.
\x
Express as a definite integral the arc length
of that part of the graph of the equation
$x = 2y - y^2$ for which $x \geq 0$.
\end{exenum}

\ex{10.2.6}
The coordinates of a particle in motion
in the plane are given by
\[
\dilemma{x=t^2,}
{y = \frac23 t^3 - \frac12 t,}
\]
at time $t$.  What is the distance which the particle
moves along its path of motion between the
time $t=0$ and $t=2$?

\ex{10.2.7}
The same curve can be defined by
more than one parametrization:
\begin{exenum}
\x
\xlab{10.2.7a}
Draw the curve defined parametrically by
\[
\dilemma{x(t) = t,}{y(t) = t, & 0 \leq t \leq 1.}
\]
\x
\xlab{10.2.7b}
Draw the curve defined parametrically by
\[
\dilemma{x(t) = \sin\pi t,}{y(t) = \sin\pi t, & 0 \leq t \leq 1.}
\]
\x
Compute the arc lengths from $t=0$ to $t=1$
for the parametrizations in \exref{10.2.7a}
and \exref{10.2.7b}.
\x
Give a geometric interpretation which explains
the difference between the arc lengths
obtained for the two parametrizations.
\end{exenum}

\ex{10.2.8}
Let $P:[a,b] \goesto \R^2$ and $Q:[c,d]\goesto R^2$
be two parametrizations of the same curve
$C$ such that all four coordinate functions
are continuously differentiable.
(A function is \dt{continuously differentiable}
if its derivative exists and is continuous at
every number in its domain.)
Then $P$ and $Q$ are called
\dt{equivalent parametrizations} of $C$
if there exists a continuously differentiable
function $f$ with domain $[a,b]$ and range $[c,d]$
which has a continuously differentiable inverse
function, and in addition satisfies

(i) $f(a) = c$ and $f(b) = d$,

(ii) $P(t) = Q(f(t)),$ for every $t$ in $[a,b]$.

\begin{exenum}
\x
Using the Chain Rule and the Change of Variable
Theorem for Definite Integrals
(for the latter, see Theorem \thref{4.6.6}),
prove that equivalent parametrizations assign
the same arc length to $C$.
\x
Show that
\[
P(t) = (\cos t, \sin t), \quad 0 \leq t \leq \frac{\pi}2
,
\]
\[
Q(s) = \left( \frac{1-s^2}{1+s^2},
\frac{2s}{1+s^2}\right), \quad 0 \leq s \leq 1
,
\]
are equivalent parametrizations of the same curve
$C$, and identify the curve.
\x
Show that
\[
P(t) = (\cos t, \sin t), \quad 0 \leq t \leq 2\pi
,
\]
and
\[
Q(s) = (\cos 5t, \sin 5t), \quad 0 \leq t \leq 2\pi
,
\]
are nonequivalent parametrizations of the circle.
\end{exenum}

\ex{10.2.9}
Prove directly from the least upper bound definition
that arc length is additive,
i.e., that ${L_a}^b + {L_b}^c = {L_a}^c$.

\end{exercises}
