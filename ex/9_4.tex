\begin{exercises}

\ex{9.4.1}
Determine whether each of the following
alternating series converges or diverges.
Give the reasons for your answers.
\begin{exenum}
\x
$\sum_{i=1}^\infty (-1)^i \frac{1}{\sqrt{i}}$
\x
$\sum_{i=1}^\infty (-1)^i \frac{1}{i^2+1}$
\x
$\sum_{k=1}^\infty (-1)^k \frac{k^2-1}{k^2+1}$
\x
$\sum_{k=1}^\infty (-1)^k \frac{1}{(k^2+1)^{\frac13}}$
\x
$\sum_{n=2}^\infty (-1)^n e^{-n}$
\x
$\sum_{i=2}^\infty (-1)^i \frac{1}{\sqrt{2i^3-1}}$
\x
$\sum_{i=0}^\infty \cos(i\pi)$
\x
$\sum_{k=1}^\infty \frac{\cos(k\pi)}{k^2}$.
\end{exenum}

\ex{9.4.2}
Prove that, for any infinite sequence
$\{ a_n \}$ of real numbers,
$\lim_{n\goesto\infty} a_n = 0$ if and only if
$\lim_{n\goesto\infty} |a_n| = 0$.
(\emph{Hint:} The proof is simple and straightforward.
Go directly to the definition of convergence
of an infinite sequence.)

\ex{9.4.3}
For each of the series
$\sum_{i=m}^\infty a_i$ in Problem \exref{9.4.1},
determine whether or not the corresponding
series of absolute values
$\sum_{i=m}^\infty |a_i|$ converges.

\ex{9.4.4}
Give an example of an alternating series
$\sum_{i=m}^\infty a_i$ which you can show
converges, but which fails to satisfy condition (i)
of the Convergence Test (\thref{9.4.1i}).

\ex{9.4.5}
The first of the following examples comes from
the formula for a geometric series,
and the last two follow from the theory
developed later in this chapter:
\begin{exenum}
\x
$\frac23 = \frac1{1+\frac12} =
\sum_{i=0}^\infty (-\frac12)^i =
1 - \frac12 + \frac14 - \cdots$.
\x
$\ln 2 = \sum_{i=1}^\infty (-1)^{i+1} \frac1i =
1 - \frac12 + \frac13 - \frac14 + \cdots$.
\x
$\pi = 4 \arctan 1 =
\sum_{i=0}^\infty (-1)^i \frac4{2i+1}=
4 - \frac43 + \frac45 - \frac47 + \cdots$.
\end{exenum}
If the value of each of these series is approximated
by a partial sum $\sum_{i=m}^\infty a_i$,
how large must $n$ be taken to ensure an error
no greater than $0.1$, $0.01$, $0.001$, $10^{-6}$?

\end{exercises}
