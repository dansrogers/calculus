\begin{exercises}

\ex{8.2.1}
For each of the following limits, find a function $f(x)$
such that the limit is equal to $\int_0^1 f(x) \; dx$.
Evaluate the limit.
\begin{exenum}
\x
$\lim_{n\goesto\infty}
\frac{1+2^2+3^2+\cdots+n^2}{n^3}$.
\x
$\lim_{n\goesto\infty}
\frac{(1+n^2)+(2^2+n^2)+(3^2+n^2)+\cdots+(n^2+n^2)}
{n^3}$.
\x
$\lim_{n\goesto\infty}
\frac{\sqrt{1+n}+\sqrt{2+n}+\sqrt{3+n}+\cdots+\sqrt{n+n}}
{n^{\frac32}}$.
\x
$\lim_{n\goesto\infty}
\frac1{\sqrt n}
\left(\frac1{\sqrt{1+n}}+\frac1{\sqrt{2+n}}+
\frac1{\sqrt{3+n}}+\cdots+\frac1{\sqrt{n+n}}
\right)$.
\end{exenum}

\ex{8.2.2}
Prove that
\begin{exenum}
\x
$\ln 2 = \lim_{n\goesto\infty}
\left( \frac1{n+1}+\frac1{n+2}+\cdots+\frac1{n+n}
\right)$.
\x
$\pi = \lim_{n\goesto\infty} \frac4{n^2}
\left(\sqrt{n^2-1}+\sqrt{n^2-2}+\cdots+\sqrt{n^2-n^2}
\right)$.
\x
$\int_1^3(x^2+1)\;dx=
\lim_{n\goesto\infty} \frac4{n^3}
\sum_{i=1}^n(n^2+2in+2i^2)$.
\x
$\frac{\pi}6 = \lim_{n\goesto\infty}
\left(\frac{1}{\sqrt{4n^2-1}}+\frac{1}{\sqrt{4n^2-2^2}}+
\cdots+\frac{1}{\sqrt{4n^2-n^2}}\right)$.
\end{exenum}

\ex{8.2.3}
Use the Trapezoid Rule with $n=4$
to compute approximations to the following
integrals.
In \exref{8.2.3a}, \exref{8.2.3b}, \exref{8.2.3c},
and \exref{8.2.3d}, compare the approximation
obtained with the true value.
\begin{exenum}
\x
\xlab{8.2.3a}
$\int_0^1 (x^2+1) \; dx$
\x
\xlab{8.2.3b}
$\int_0^2 (x^2+1) \; dx$
\x
\xlab{8.2.3c}
$\int_{-1}^3 (4x-1) \; dx$
\x
\xlab{8.2.3d}
$\int_1^3 \frac1{x^2} dx$
\x
$\int_0^1 \frac1{1+x} dx$
\x
$\int_0^1 \frac{dx}{1+x^2}$
\x
$\int_0^1 e^{-x^2} dx$
\x
$\int_0^1 \frac1{x^2+x+1} dx$
\x
$\int_0^1 \frac{x^2-1}{x^2+1} dx$
\x
$\int_0^{\pi} \frac{\sin x}{x} dx$.
\end{exenum}

\ex{8.2.4}
Show geometrically, without appealing to
Theorem \thref{8.2.4}, that the approximation
$T_n$ obtained with the Trapezoid Rule
has the following properties.
\begin{exenum}
\x
If $f^{\prime\prime}(x) \geq 0$
for every $x$ is $[a,b]$, then
$T_n \geq \int_a^b f$.
\x
If $f^{\prime\prime}(x) \leq 0$
for every $x$ in $[a,b]$, then
$T_n \leq \int_a^b f$.
\end{exenum}

\ex{8.2.5}
For each of the following integrals,
use Theorem \thref{8.2.4} as the basis
for finding the smallest integer $n$
such that the error
$|\int_a^b f - T_n|$ in applying the
Trapezoid Rule is less than
(i) $\frac1{100}$, (ii) $\frac1{10000}$,
and (iii) $10^{-8}$.
\begin{exenum}
\x
$\int_1^4 \frac1{6x^2} dx$
\x
$\int_0^1 (8x^3-5x+3) \; dx$
\x
$\int_{-1}^2 (3x+1)\;dx$
\x
$\int_1^2 \frac1x dx$
\x
$\int_0^{12} \frac1{16x+2}dx$
\x
$\int_0^1 e^{-x^2} dx$.
\end{exenum}

\ex{8.2.6}
(For those who have access to a
high-speed digital computer and know how to use it.)
Compute the Trapezoid approximation $T_n$
to each of the following integrals.
\begin{exenum}
\x
$\int_0^1 \frac1{1+x^3} dx$,
for $n=10$, $100$, and $1000$.
\x
$\int_0^1 \frac1{1+x^2} dx$,
for successive values of $n = 10,100,1000,\ldots$,
until the error is less that $10^{-6}$.
\x
$\int_0^1 \sqrt{1+x^3} \; dx$,
for $n = 5$, $50$, and $500$.
\x
$\int_0^{\pi} \frac{\sin x}x dx$,
for $n = 2$, $4$, $8$, $16$, and $100$.
\x
$\int_0^1 e^{-x^2}dx$,
for $n=10$, $100$, and $1000$.
\end{exenum}

\end{exercises}
