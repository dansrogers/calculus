\begin{exercises}
\ex{1.1.1}
Draw the following intervals and identify them as bounded or unbounded,
closed or open, or neither:
$(2, 4)$, $[3, 5]$, $(-\infty, -2]$, $[1.5, 2.5)$, $(\sqrt2, \pi)$.

\ex{1.1.2}
Draw each of the following subsets of $R$.
For those that are given in terms of absolute values write an alternative description
that does not use the absolute value.
\begin{exenum}
\sx
Set of all $x$ such that $4 < x \leq 7.5$.
\sx
Set of all $x$ such that $0 < x < \infty$.
\sx
Set of all $x$ such that $5 \leq x < 8$.
\sx
Set of all $x$ such that $|x| > 2$.
\sx
Set of all $y$ such that $1 < |y| < 3$.
\sx
Set of all $z$ such that $|z - 2| \leq 1$.
\sx
Set of all $x$ such that $|x - a| > 0$.
\sx
Set of all $u$ such that $1 < |u - 1| < 5$.
\end{exenum}

\ex{1.1.3}
Prove the following facts about inequalities.
[\emph{Hint:}\ Use \ref{axiom.viii}, \ref{axiom.ix},
\ref{axiom.x}, \thref{1.1.1},
and the meanings of $\geq$ and $\leq$.
In each problem you will have to consider several cases separately,
e.g. $a > 0$ and $a = 0$.]
\begin{exenum}
\sx
If $a \leq b$, then $a + c \leq b + c$.
\sx
If $a \geq b$, then $a + c \geq b + c$.
\sx
If $a \leq b$ and $c \geq 0$, then $ac \leq bc$.
\sx
If $a \leq b$ and $ c \leq 0$, then $ac \geq bc$.
\end{exenum}

\ex{1.1.4}
Prove that $a$ is positive (negative) if and only if $\frac1a$ is positive (negative).

\ex{1.1.5}
If $0 < a < b$, prove that $\frac1b < \frac1a$.

\ex{1.1.6}
If $a > c$ and $b < 0$, prove that $\frac ab < \frac cb$.

\ex{1.1.7}
If $a < b < c$, prove that
\[
\cond{\frac bc < \frac ba & \mbox{if $a > 0$}}
,
\]
\[
\cond{\frac bc > \frac ba & \mbox{if $c < 0$}}.
\]

\ex{1.1.8}
Does the set $Z$ of integers have the Least Upper Bound Property?
That is, if a nonempty subset of $Z$ has an upper bound,
does it have a smallest one?

\ex{1.1.9}
Show that if $0 \leq a \leq b$, then $0 \leq \sqrt{a} \leq \sqrt{b}$.

\ex{1.1.10}
Prove that $a = b$ if and only if $a \leq b$ and $b \leq a$.

\ex{1.1.11}
Show that the Least Upper Bound Property implies the Greatest Lower Bound Property.
That is, using \ref{axiom.xi}, prove that if a nonempty subset of $R$ has a lower bound,
then it has a greatest lower bound.

\ex{1.1.12}
Verify the assertion made in the text that if an interval is bounded it must be one of four types:
$(a, b)$, $[a, b]$, $(a, b]$, or $[a, b)$.  (\emph{Hint:} See Problem \exref{1.1.11}.)

\ex{1.1.13}
Prove that $\sqrt2$ is irrational.
(\emph{Hint:} The proof, which is elegant and famous, starts by assuming that 
$\sqrt2 = \frac pq$, where $p$ and $q$ are integers not both even.
A contradiction can then be derived.)

\end{exercises}
