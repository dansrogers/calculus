\begin{exercises}

\ex{4.8.1}
A straight highway connects towns $A$ and $B$.
A car starts at $t = 0$ from $A$ and goes toward
$B$ with a velocity given by $v(t) = 60t - 12t^2$,
measured in miles per hour.
When the car arrives at $B$,
it is slowing down and its speed is $48$ miles per hour.
\begin{exenum}
\x
How far apart are the two towns?
\x
What are the maximum and minimum
speeds obtained during the trip?
When are they reached,
and at what distances?
\end{exenum}

\ex{4.8.2}
A straight highway connects towns $A$ and $B$.
A car, initially stopped, starts at $t = 0$ from $A$
and accelerates at $240$ miles per hour per hour until
reaching a speed of $60$ miles per hour.
\begin{exenum}
\x
How long does this take, both in time and distance?
Assume that the car travels at the constant speed of
$60$ miles per hour once it has reached that speed,
and that is slows down to a stop at town $B$ in
the same way that it left $A$.
\x
How far apart are $A$ and $B$ if the whole
trip takes $5$ hours?
\end{exenum}

\ex{4.8.3}
A projectile is fired straight up with an initial velocity
of $640$ feet per second
(see Example \exampref{4.8.2}).
\begin{exenum}
\x
Find the velocity $v(t)$.
\x
How far does the projectile travel during the first
$10$ seconds of its flight?
\x
How far does the projectile go,
and how many seconds after takeoff is this
maximum height reached?
\x
What is the total distance traveled by the projectile
during the first $30$ seconds
of its flight?
\x
What is the velocity when the projectile
returns to the ground?
\end{exenum}

\ex{4.8.4}
Let the function $f$ be integrable over the
interval $[a,b]$, and suppose that $f(x)$
does not change sign on the interval.
Prove that
\[
\int_a^b |f(x)| \; dx = \left| \int_a^b f(x) \; dx \right|
.
\]
(This is an easy problem.
Consider separately the two cases:
First, $f(x) \geq 0$ for every $x$ in $[a,b]$,
and second, $f(x) \leq 0$ for every $x$
in $[a,b]$.)

\ex{4.8.5}
A particle moves on the $x$-axis with velocity
given by $v(t) = -4t + 20$.
\begin{exenum}
\x
In which direction is the particle moving at time $t = 0$?
\x
Find $s(t)$, the position of the particle at time $t$,
if its coordinate is $-30$ when $t = 1$.
\x
Find the distance traveled by the particle during
the time interval from $t = 0$ to $t = 4$.
\x
Find the distance traveled by the particle during the
time interval from $t = 0$ to $t = 8$.
\x
When is $s(t) = 0$?
\end{exenum}

\ex{4.8.6}
A particle moves on the $y$-axis with acceleration
given by $a(t) = 6t - 2$.
Denote its velocity and position at time $t$ by $v(t)$
and $y(t)$, respectively.
At time $t = 1$, the particle is at rest at the zero
position.
\begin{exenum}
\x
Find $v(t)$ and $y(t)$.
\x
How far does the particle move during the
time interval from $t = 1$ to $t = 3$?
\x
What is the distance traveled by the particle
from $t = -1$ to $t = 2$?
\end{exenum}

\ex{4.8.7}
A road borders a rectangular forest,
and a car is driven around it.
The car starts from rest at one corner and accelerates
at $120$ miles per hour per hour until it reaches
the next corner $15$ minutes later.
The second side is $20$ miles long and the
car is driven along it at constant velocity
equal to the final velocity reached on the first side.
The car continues at this same speed on the third side.
On the fourth side, however, the car slows down
with constant acceleration and comes to a stop
at its original starting place.  Find
\begin{exenum}
\x
the dimensions of the rectangle.
\x
the acceleration on the fourth side.
\x
the time taken for the whole trip.
\end{exenum}

\ex{4.8.8}
Let the function $f$ be integrable over the interval
$[a,b]$.  From the definition of integrability in Section
\secref{4.1}, prove that $\int_a^b f$ is the only
number such that
\[
L_\sigma \leq \int_a^b f \leq U_\sigma
,
\]
for every partition $\sigma$ of $[a,b]$.

\ex{4.8.9}
A conical funnel of height $36$ inches and base
with radius $12$ inches is initially filled with sand.
At $t = 0$, the sand starts running out the bottom
(apex of the cone) so that the volume $V$ of sand
remaining in the funnel is decreasing at the constant
rate of $10$ cubic inches per minute.
\begin{exenum}
\x
Find $V$ as a function of time $t$, and determine
how long it takes for all the sand to run out.
\x
Assuming that the sand retains its original conical
shape during the process, find the radius $r$
of the base of the cone of sand as a function of $t$.
\end{exenum}

\ex{4.8.10}
A particle moves on the parabola $y = x^2$,
and its horizontal component of velocity is given
by $x^\prime (t) = \frac1{(t + 1)^2}$, $t \geq 0$.
At time $t = 0$ the particle is at the origin.
\begin{exenum}
\x
What are the $x$ and $y$ coordinates of the particle
when $t = 1$?
When $t = 3$?
\x
As $t$ increases without bound what happens
to the particle?
\end{exenum}

\ex{4.8.11}
At time $t = 0$, an object is dropped from an airplane
which is moving horizontally with velocity $v_0$.
Its downward acceleration $y^{\prime\prime}(t)$
is constant and equal to $-g$.
Measure the positive direction of $x$ in the direction
of motion of the airplane and the positive direction
of $y$ upward.  Also assume that
$x(0) = y(0) = 0$.
\begin{exenum}
\x
\xlab{4.8.11a}
Find $x(t)$ and $y(t)$.
\x
By eliminating $t$ from the equations in \exref{4.8.11a},
find the equation in terms of $x$ and $y$ in which
the object falls.
What is the name of the curve?
\end{exenum}

\end{exercises}
