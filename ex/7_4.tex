\begin{exercises}

\ex{7.4.1}
Separate each of the following into the sum of a
polynomial and a sum of partial fractions.
\begin{exenum}
\x
$\frac{5}{(x-2)(x+3)}$
\x
$\frac{x+2}{(2x+1)(x+1)}$
\x
$\frac{2x^3+3x^2-2}{2x^2+3x+1}$
\x
$\frac{4x^2-5x+10}{(x-4)(x^2+2)}$
\x
$\frac{3x^3+5x^2-27x+8}{x^2+4x}$
\x
$\frac{x^2+1}{x^2+x+1}$.
\end{exenum}

\ex{7.4.2}
Integrate each of the following.
\begin{exenum}
\x
$\int \frac{5}{(x-2)(x+3)} \; dx$
\x
$\int \frac{x+2}{(2x+1)(x+1)} \; dx$
\x
$\int \frac{2x^3+3x^2-2}{2x^2+3x+1} \; dx$
\x
$\int \frac{4x^2-5x+10}{(x-4)(x^2+2)} \; dx$.
\end{exenum}

\ex{7.4.3}
Find the partial fractions decomposition of each
of the following rational functions.
\begin{exenum}
\x
$\frac{x-8}{x^2-x-6}$
\x
$\frac{18}{x^2+8x+7}$
\x
$\frac{x+1}{(x-1)^2}$
\x
$\frac{8x+25}{x^2+5x}$
\x
$\frac{4}{x^2(x+2)}$
\x
$\frac{6x^2-x+13}{(x+1)(x^2+4)}$
\x
$\frac{(x+2)^2}{(x+3)^3}$
\x
$\frac{x^2+2x+5}{(2x-1)(x^2+1)^2}$.
\end{exenum}

\ex{7.4.4}
Evaluate each of the following integrals.
\begin{exenum}
\x
$\int \frac{x-8}{x^2-x-6} \; dx$
\x
$\int \frac{18}{x^2+8x+7} \; dx$
\x
$\int \frac{x+1}{(x-1)^2} \; dx$
\x
$\int \frac{8x+25}{x^2+5x} \; dx$
\x
$\int \frac{6x^2-x+13}{(x+1)(x^2+4)} \; dx$
\x
$\int \frac{(x+2)^2}{(x+3)^3} \; dx$.
\end{exenum}

\ex{7.4.5}
\begin{exenum}
\x
\xlab{7.4.5a}
Show directly that $\frac{2x-3}{(x-2)^2}$
can be written in the form
$\frac{A}{x-2} + \frac{B}{(x-2)^2}$
by first writing
$\frac{2x-3}{(x-2)^2} =
\frac{2(x-2)+1}{(x-2)^2}$.
\x
\xlab{7.4.5b}
Following the method in \exref{7.4.5a},
show that $\frac{ax+b}{(x-k)^2}$ can always be
written $\frac{A}{x-k} + \frac{B}{(x-k)^2}$,
where $A$ and $B$ are constants.
\x
Extend the result in \exref{7.4.5b} by factoring,
completing the square, and dividing to show
directly that
\[
\frac{ax^2+bx+c}{(x-k)^3}
\: \mbox{can be written} \:
\frac{A}{x-k}+\frac{B}{(x-k)^2}+\frac{C}{(x-k)^3}
\]
where $A$, $B$ and $C$ are constants.

[\emph{Note:} Without knowledge of the algebraic
theory of partial fractions, it would not be
unreasonable to assume that a decomposition
of a rational function $\frac{N(x)}{P(x)(x-k)^3}$
would necessarily contain fractions
$\frac{A}{x-k}$, $\frac{Bx+C}{(x-k)^2}$,
and $\frac{Dx^2+Ex+F}{(x-k)^3}$.
This problem shows, however, that in the complete
decomposition $B = D = E = 0$.]
\end{exenum}

\ex{7.4.6}
Why can there not be an irreducible cubic
polynomial with real coefficients?

\ex{7.4.7}
Integrate each of the following.
\begin{exenum}
\x
$\int \frac{(3x+1)\;dx}{x^3+2x^2+x}$
\x
$\int \frac{(x^2+1)\;dx}{x^2-3x+2}$
\x
$\int \frac{(x-2)\;dx}{(2x+1)(x^2+1)}$
\x
$\int \frac{x^2-3x-2}{(x-2)^2(x-3)}\;dx$
\x
$\int \frac{dx}{x^2+2x+2}$
\x
$\int \frac{(2x+1)\;dx}{x^2+2x+2}$
\x
$\int \frac{\sec^2x\;dx}{\tan^2x-4\tan x+3}$
\x
$\int \frac{\sec y \tan y \; dy}{2\sec^2y+5\sec y+2}$
\x
$\int \frac{y^2+1}{y^2+y+1}\;dy$
\x
$\int \frac{10+5z-z^2}{(z+4)(z^2+z+1)}\;dz$
\x
$\int \frac{(6x+3)\;dx}{(x-1)(x+2)(x^2+x+1)}$
\x
$\int \frac{(x^3+4)\;dx}{(x+1)(x+2)^2}$
\x
$\int \frac{x^2+2x+5}{(2x-1)(x^2+1)^2}\;dx$
\x
$\int \frac{dx}{(x^2+x+5)^3}$.
\end{exenum}

\ex{7.4.8}
Prove that the statement in the text, \secref{7.4} that,
since a nonzero polynomial of degree $n$
has at most $n$ distinct roots, two rational
functions with the same denominator are
equal if and only if their numerators are equal.
[\emph{Hint:} Suppose that
$\frac{N_1(x)}{D(x)} = \frac{N_2(x)}{D(x)}$,
where polynomial $D(x)$ is not the zero function.
Then $\frac{N_1(x)-N_2(x)}{D(x)}=0$,
and so the polynomial equation
$N_1(x) - N_2(x) = 0$ holds for every real number
$x$ for which $D(x) \ne 0$.]

\end{exercises}
