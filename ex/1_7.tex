\begin{exercises}

\ex{1.7.1}
With the aid of the rules for differentiation given in this section,
compute $f^\prime = \ddxof f$ for each of the following functions.
\begin{exenum}
\sx
$f(x) = 3x^2 + 4x + 1$
\sx
$f(x) = x^2 (x + 1)$
\sx
$f(x) = x^3 (x + 2)^2$
\sx
$f(x) = (x^2 - 4)(x^2 + 2x + 3)$
\sx
$f(x) = 2x^2 + \frac1{3x^3}$
\sx
$f(x) = \frac{2x}{2-x}$
\sx
$f(x) = \frac{2x}{(2-x)^2}$
\sx
$f(x) = \frac{x^3}{x^5 + 1}$
\sx
$f(x) = \left( \frac{3-x}{3+x} \right)^2$
\sx
$f(x) = (x^2 + 1)^3$
\sx
$f(x) = \frac{2x+1}{x^2+x}$
\sx
$f(x) = (x^2 + 1)^{-1}$
\sx
$f(x) = (x + x^{-1})^2$
\sx
$f(x) = (x-a)(x-b)(x-c)$
\end{exenum}

\ex{1.7.2}
Determine an equation of the line tangent to the parabola
$y = x^2 - 4x + 5$ at the point $(1,2)$.
Draw the parabola and the tangent line.

\ex{1.7.3}
The parabola $y = ax^2 + bx + c$ passes through $(0,4)$
and is tangent to the line $2x + y = 2$ at the point $(1, 0)$,
Find the coefficients $a$, $b$, and $c$ for the parabola.

\ex{1.7.4}
Show that if $f$, $g$, and $h$ are differentiable functions, then
\[
(fgh)^\prime = f^\prime gh + fg^\prime h + fgh^\prime
.
\]

\ex{1.7.5}
What is the correct product rule for differentiation, analogous
to the one in Problem \exref{1.7.4}, for (a) four factors, (b) $n$ factors?

\ex{1.7.6}
Obtain an equation of the tangent line to the graph of the function
$f(x) = \frac{x^3}{x^2 + 1}$ at the point where $x = 2$.

\ex{1.7.7}
\begin{exenum}
\sx
If $f(z) = 2z^2 + 2 + \frac2{z^2}$, then $f^\prime (2) = \cdots$.
\sx
If $f(z) = 2z^2 +2 + \frac2{z^2}$, then $f^\prime (x) = \cdots$.
\sx
If $y = \frac{x+1}{x-1}$, then $\dydx = \cdots$.
\sx
If $y = \frac1x$, then $\dydx (2) = \cdots$.
\sx
If $f(x) = \frac{x^2 + 1}{x^2}$, then $\ddxof f (a) = \cdots$.
\sx
If $w = 3u^2 + 4u + 2$, then $\nxder{}{w}{u} = \cdots$.
\end{exenum}

\ex{1.7.8}
The parabola $y = ax^2 + bx + c$ is tangent to the line
$y = 4x + 7$ at the point $(-1, 3)$.
In addition, $\dydx (-2) = 0$.  Find the coefficients $a$, $b$, and $c$.

\ex{1.7.9}
For each of the following functions $f$, compute the derivative
$f^\prime$ and the second derivative $f^{\prime\prime}$.
\begin {exenum}
\sx
$f(x) = 3x^2 + 2x + 1$
\sx
$f(x) = 5x + 1$
\sx
$f(x) = \frac{x^4}{12} + \frac{x^3}6 + \frac{x^2}2 + x + 1$
\sx
$f(t) = t^3(t^2 - 1)$
\sx
$f(x) = x^3 + \frac1{x^2}$
\sx
$f(s) = \frac{s^2 - 1}{s^2 + 1}$.
\end{exenum}

\ex{1.7.10}
The line $y = 3x - 1$ is tangent to the graph of the function
$f(x) = ax^3 + bx^2 + c$ at the point $(1,2)$.
Furthermore, $\deriv{2}{f} (1) = 0$.  Compute $a$, $b$, and $c$.

\ex{1.7.11}
\begin{exenum}
\sx
If $f(x) = x^3 - x^2 +x -1$, then $\deriv{2}f = \cdots$.
\sx
If $y = \frac{x-1}{x+1}$, then $\deriv2y = \cdots$.
\sx
If $s = at^3 + bt^2 + ct + d$, where $a$, $b$, $c$, and $d$ are constants,
compute $\nxder3st$.
\sx
If $y = \frac1{x^2}$, then $\deriv3y (a) = \cdots$.
\end{exenum}

\ex{1.7.12}
Find all the points on the graph of the function $\frac{x^3}3 - x^2$ at
which the tangent line is perpendicular to the tangent line at $(1, -\frac23)$.

\ex{1.7.13}
There are many examples of a function $f$ and a number $a$
such that $f(a)$ is defined ($a$ is in the domain of $f$) but
$f^\prime(a)$ does not exist.  Another way of saying the same thing
is that the domain of $f^\prime$ can be a {\emph proper} subset
of the domain of $f$.  It is equally possible for $f^\prime (a)$ to be
defined and $f^{\prime\prime} (a)$ not to be.  Let $f$ be the function
defined by
\[
f(x) = \dilemma{\frac{x^2}2 & \mbox{if $x\geq 0$,}}
{-\frac{x^2}2 & \mbox{if $x\leq 0$.}}
\]
\begin{exenum}
\sx
Compute $f^\prime$.
\sx
Is $f$ a differentiable function?
[That is, does $f^\prime(a)$ exist for every real number $a$?]
\sx
Show that $f^{\prime\prime}(0)$ does not exist,
and compute $f^{\prime\prime}(x)$ for $x\ne 0$.
\end{exenum}

\ex{1.7.14}
Same as Problem \exref{1.7.13} except that $f(x) = x^{\frac43}$.

\ex{1.7.15}
\begin{exenum}
\sx
Draw the graph of the function $g$ defined by
\[
g(x) = \dilemma{x^2, & x \leq 1,}
{2x-1, & x > 1.}
\]
\sx
Compute $g^\prime$ and $g^{\prime\prime}$.
\sx
Are $g$ and $g^\prime$ differentiable functions?
\end{exenum}

\end{exercises}
