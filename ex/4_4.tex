\begin{exercises}

\ex{4.4.1}
Expand each of the following integrals.
That is, write each one as a sum of constant multiples
of the integrals of the powers of the variables.
\begin{exenum}
\x
$\int_0^1 (x^2 + 5x) \; dx$
\x
$\int_2^3 (4x^5 - x - 2) \; dx$
\x
$\int_1^2 (3t^2 + 2t^2 + t) \; dt$
\x
$\int_5^3 (17y^{13} - 11y^7 + 4) \; dy$
\x
$\int_0^1 (x^2 + 2)^2 \; dx$.
\end{exenum}

\ex{4.4.2}
Given that $\int_0^1 x^n \; dx = \frac1{n+1}$,
for every nonnegative integer $n$, evaluate
\begin{exenum}
\x
$\int_0^1 (2x^2 + 3x) \; dx$
\x
$\int_0^1 (5x^3 - x^2 - 2) \; dx$
\x
$\int_0^1 (3t^2 - 1) \; dt$
\x
$\int_0^1 (x + 2)^2 \; dx$
\x
$\int_0^1 (3y^2 - y + 1) \; dy$.
\end{exenum}

\ex{4.4.3}
Use the result
\[
\int_1^2 x^n \; dx = \frac{2^{n+1} - 1}{n+1},
\quad n=0,1,2,\ldots
,
\]
and the analogous result at the beginning of
Problem \exref{4.4.2} to evaluate
\begin{exenum}
\x
$\int_1^2 (3x^2 - 2x + 1) \; dx$
\x
$\int_0^2 x^2 \; dx$
\x
$\int_0^2 (4x^3 - 3x + 2) \; dx$
\x
$\int_0^2 (t^3 + t^2 + t) \; dt$.
\end{exenum}

\ex{4.4.4}
Using the definition of integrability,
prove Theorem \thref{4.4.4}.
(\emph{Suggestion:} Treat the cases
$k \geq 0$ and $k \leq 0$ separately.)

\ex{4.4.5}
Using \thref{4.4.3}, prove that if $f$ and $g$ are
integrable over $[a,b]$ and if $f(x) = g(x)$,
for every $x$ in $[a,b]$, then
\[
\int_a^b f(x) \; dx = \int_a^b g(x) \; dx
.
\]

\ex{4.4.6}
Prove that if $f$ is integrable over $[a,b]$ and if
$f(x) \leq M$ for all $x$ in $[a,b]$, then
\[
\int_a^b f(x) \; dx \leq M(b-a)
.
\]

\ex{4.4.7}
Replace the symbol $*$ by either $\leq$ or $\geq$
so that the resulting expressions are correct.
Give your reasons.
\begin{exenum}
\x
$\int_0^1 x^2 \; dx * \int_0^1 x^3 \; dx$
\x
$\int_{-1}^1 x^2 \; dx * \int_{-1}^1 x^3 \; dx$
\x
$\int_1^3 x^2 \; dx * \int_1^3 x^3 \; dx$.
\end{exenum}

\ex{4.4.8}
Plot the graph of the function $f(x) = 1 - x^2$,
and indicate the region $P^+$ defined by the inequalities
$0 \leq x \leq 2$ and $0 \leq y \leq f(x)$ and the region $P^-$
defined by the inequalitiy $0 \leq x \leq 2$
and $f(x) \leq y \leq 0$.
\begin{exenum}
\x
Use the identities given in Problems {4.4.2} and {4.4.3}
to evaluate the integrals
$\int_0^1 f(x) \; dx$, $\int_1^2 f(x) \; dx$,
and $\int_0^2 f(x) \; dx$.
\x
Find $\mathit{area}(P^+)$, $\mathit{area}(P^-)$,
and $\mathit{area}(P^+ \cup P^-)$.
\end{exenum}

\ex{4.4.9}
Draw the graph of the function
$f(x) = x(x-2)(x-4) = x^3 - 6x^2 + 8x$,
and indicate the region $P^+$  defined by the inequalities
$0 \leq x \leq 3$ and $0 \leq y \leq f(x)$, and the region
$P^-$ defined by $0 \leq x \leq 3$ and $f(x) \leq y \leq 0$.
Let $P = P^+ \cup P^-$, and suppose that
$\int_0^2 f(x) \; dx = 4$ and $\int_0^3 f(x) \; dx = 2\frac14$.
Find $\mathit{area}(P^+)$, $\mathit{area}(P^-)$,
and $\mathit{area}(P)$.

\ex{4.4.10}
Prove case \thref{4.4.7}(iii)   of Theorem \thref{4.4.7}.

\ex{4.4.11}
Consider a function $f$ which is integrable over
$[a,b]$ and which, in addition, satisfies:

(i)$f$ is continuous at every point of $[a,b]$.

(ii)$f(x) \geq 0$, for every $x$ in $[a,b]$.

(iii)$f(c) > 0$ for at least one point $c$ in $[a,b]$.

Prove that $\int_a^b f(x) \; dx > 0$.

\end{exercises}
