\begin{exercises}

\ex{1.5.1}
For each of the following lines, find an equation that defines it.
\begin{exenum}
\sx
The line through $(2,3)$ with slope $1$.
\sx
The line through $(0,1)$ with slope $1$.
\sx
The line through $(0,1)$ with slope $-2$.
\sx
The line through $(-1, -3)$ with slope $-\frac12$.
\sx
The line through $(-2, 1)$ and $(-1,-1)$.
\sx
The line containing the point $(1, 0)$ and $(0,1)$.
\sx
The line through the origin containing the point $(1, -19)$.
\sx
The line with slope $0$ that passes through $(3, 4)$.
\sx
The line through $(2, 5)$ and $(2, 8)$.
\end{exenum}

\ex{1.5.2}
Draw the line defined by each of the following equations, and find the slope.
\begin{exenum}
\sx
$x+y=1$
\sx
$x=-y$
\sx
$2x-4y=3$
\sx
$7x=3$
\sx
$7y=3$
\sx
$4x + 3y = 10$.
\end{exenum}

\ex{1.5.3}
Determine whether $P$, $Q$, and $R$ lie on a line.
If they do, draw the line and write an equation for it.
\begin{exenum}
\sx
$P = (0,0)$, $Q = (-1, 3)$, $R = (3, -4)$.
\sx
$P = (\frac12,\frac32)$, $Q = (\frac52, -\frac72)$, 
$R = (-\frac32, -\frac{13}2)$.
\sx
$P = (a_1,a_2)$, $Q = (b_1, b_2)$, $R = (c_1, c_2)$.
\end{exenum}

\ex{1.5.4}
Draw the set of all ordered pairs $(x,y)$ such that
\begin{exenum}
\sx
$4x^2 + 4xy + y^2 + 12x + 6y + 9 = (2x + y + 3)^2 = 0$.
\sx
$5x^2 + 7xy + 2y^2 + 3x + 3y = (5x + 2y + 3)(x + y)= 0$.
\end{exenum}

\ex{1.5.5}
The $x$-coordinate of a point where a curve intersects the $x$-axis is called
an \dt{$x$-intercept} of the curve.
The definition of a \dt{$y$-intercept} is analogous.
\begin{exenum}
\sx
Find the $x$- and $y$-intercept of the line defined by $y-3x=10$.
Draw the line.
\sx
Write an equation for the line with slope $m$ and $y$-intercept equal to $b$.
\end{exenum}

\ex{1.5.6}
For each of the following equations, define the function $f(x)$
whose graph is the set of ordered pairs that satisfy the equation.
Which ones are linear functions?
\begin{exenum}
\sx
$3x-y=7$
\sx
$5y=3$
\sx
$2|x| + 3y= 4$
\sx
$x-y=1$
\sx
$y^2 + 2x +3=0$ (two functions)
\sx
$x^2 - 2xy + y^2 = 0$
\sx
$y = 3x^2 + 4x +2$
\sx
$5x + 3y = 1$.
\end{exenum}

\ex{1.5.7}
Among the lines defined by the following equations,
which pairs are parallel and which perpendicular?
\begin{exenum}
\sx
$4x + 2y = 13$
\sx
$3x - 6y = 0$
\sx
$3x + 2y = 6$
\sx
$y = -2x$
\sx
$4x = 13$
\sx
$4y = 13$.
\end{exenum}

\ex{1.5.8}
\begin{exenum}
\sx
Write an equation of the straight line $L_1$ that contains the points
$(1,3)$ and $(3, -2)$.
\sx
Write an equation of the line with $x$-intercept $1$
that is parallel to $L_1$.
\sx
Write an equation of the line perpendicular to $L_1$ that passes
through $(1, 3)$.
\end{exenum}

\ex{1.5.9}
Prove that the two lines $L_1$ and $L_2$ in Figure \figref{1.32} are
perpendicular.  (\emph{Hint:} Use congruent right triangles
or the converse of the Pythagorean Theorem.)


\end{exercises}
