\section{Sequences and Summations.} We shall return to the definite integral in Section 3. The purpose of the present digression is to develop some techniques, applicable not only to the study of the integral but also to many other parts of mathematics.

Most of the functions studied in this book have as domains intervals on the real line, or unions of intervals; e.g., the domain of the function $\frac{1}{x}$ is the union $(-\infty, 0) \cup (0, \infty)$. In this section, on the other hand, we are concerned with functions whose domains are sets of integers. An example is the function a defined by $a(n) = \sqrt{n - 2}$, for every integer greater than 1. If $a$ is a function whose domain is a subset of the integers, it is common practice to denote its value at $n$ by $a_n$. Thus

$$
a_n = a(n).
$$
%SEC. 2] SEQUENCES AND SUMMATIONS  175
\noindent A simple example in which the domain is a finite set of integers is a partition of an interval in which we have indexed the points of the partition as $x_0, . . ., x_n$. In this case,
$$
x_i = x(i), \;\;\; \mbox{for}\; i = 0, . . ., n.
$$

We come next to the definition of a sequence, which is a special case of a function defined on a set of integers. We shall accept the intuitive idea of a sequence to be that of a list (in mathematics, most likely, a list of numbers). With this in mind, we define a \textbf{sequence} to be a function whose domain $D$ is a set of integers such that
\medskip


\begin{description}
\item [(i)] $D$ is a set of consecutive integers; i.e., if $i$ and $j$ are in $D$, 
then every integer between $i$ and $j$ is also in $D$.
\item [(ii)] $D$ contains a least element.
\end{description}

If $s$ is a sequence and if $l$ is the least, or smallest, integer in its domain, then $s(l) = s_{l}$ is the first member of the sequence, $s(l + 1) = s_{l + 1}$ is the second member, and so on. In the most common applications $l$ is either 0 or 1, and so the values of the sequence appear as either $s_0, s_1, s_2,...$ or as $s_1, s_2, ...$.

A sequence is \textbf{finite} or \textbf{infinite} according as its domain $D$ is finite or infinite. Note that the range of an infinite sequence need not contain infinitely many numbers. The function $a$ defined, for every positive integer $n$, by

\begin{equation}
a_{n} = a(n) = \left\{  \begin{array}{ll}
                                 0, & \mbox{if $n$ is even,} \\
                                 1, & \mbox{if $n$ is odd,}
                                 \end{array}
                       \right.
\label{eq4.2.1}
\end{equation}
\noindent is the infinite sequence 1, 0, 1, 0, 1, 0, 1, .... An even simpler example of an infinite sequence is the constant function $b$ defined by
$$
b_n = 1, \;\;\;\mbox{for every positive integer}\; n.
$$
A common notation for a sequence $s$, whether finite or infinite, is $\{ s_n \}$. When a sequence is written in this way, the letter $n$ is called an \textbf{index}.  Like the variable of integration in a definite integral, it is a dummy symbol. Any letter can be used, although $n, m, i, j$, and $k$ are the most common. Thus
$$
s = \{ s_n \} = \{ s_m \} = \{ s_i \} = \mbox{etc.}
$$
Of course, a finite sequence can be described by simply enumerating its terms, e.g., $s_1, . . ., s_n$, 
or $a_3, a_4, . . ., a_{10}$.

We shall study two major topics in this section. The first is the limit of an infinite sequence.  This is actually just an application of the idea of the limit of a function which we defined and studied in Chapter 1. As an example, let s be the infinite sequence defined by
$$
s_n = \frac{2n^2 + n - 1}{3n^2 - 2n + 2}, \;\;\;  \mbox{for every positive integer}\; n.
$$
%t76 INTEGRA7ElON [CHAP. 4
\noindent We ask for the limit of $\{ s_n \}$ as $n$ increases without bound, which we denote by $\lim_{n \rightarrow \infty} s_n$. Dividing both numerator and denominator of the above expression by $n^2$, we obtain 

$$
s_n = \frac{2 + \frac{1}{n} - \frac{1}{n^2}}{3 - \frac{2}{n} + \frac{2}{n^2}}.
$$
\noindent If $n$ is very large, it is clear that $2 + \frac{1}{n} - \frac{1}{n^2}$ is nearly equal to 2, and that $3 - \frac{2}{n} + \frac{2}{n^2}$ is nearly equal to 3. We conclude that the number which the values of the sequence are approaching, i.e., the limit, is $\frac{2}{3}$. Thus we write

$$
\lim_{n \rightarrow \infty} S_n = \lim_{n \rightarrow \infty}  \frac{2n^2 + n - 1}{3n^2 - 2n + 2} = \frac{2}{3}.
$$

%EXAMPLE 1. 
\begin{example}
Let $\{s_n \}$ and $\{a_m\}$ be two infinite sequences defined, respectively, by
$$
\begin{array}{ll}
s_n  = \frac{\sqrt{2n - 5}}{\sqrt{5n - 2}}, \;\;\; &\mbox{for}\;\;\; n=3, 4, 5, ... ,\\
a_m = \frac{m^2 + 1}{m},                      \;\;\; &\mbox{for}\;\;\; m = 1, 2, 3, ... . \end{array}
$$
\noindent Find $\lim_{n \rightarrow \infty} s_n$ and $\lim_{m \rightarrow \infty} a_m$. For the sequence $s$, we divide numerator and denominator by $\sqrt{n}$, getting

$$
s_{n} = \frac{\frac{1}{\sqrt n} \sqrt{2n-5}}{\frac{1}{\sqrt n} \sqrt{5n-2}} 
= \frac{\sqrt{2 - \frac{5}{n}}}{\sqrt{5 - \frac{2}{n}}}.
$$
\noindent Both $\frac{5}{n}$ and $\frac{2}{n}$ obviously approach 0 as a limit as $n$ increases without bound. We conclude that 

$$
\lim_{n \rightarrow \infty} s_n = \lim_{n \rightarrow \infty} \frac{\sqrt{2n - 5}}{\sqrt{5n - 2}} = \sqrt{\frac{2}{5}}.
$$
\noindent For the sequence $\{ a_m \}$ we have

$$
a_m = \frac{m^2 + 1}{m} = m + \frac{1}{m}.
$$
\noindent It is obvious that, as $m$ increases without bound, so does $m + \frac{1}{m}$. Hence no limit exists. On the other hand, we can unambiguously express the fact that
%sec. 2| SEQUENCES AND SUMMATIONS  177
the values of the sequence are increasing without bound by writing
$$
\lim_{m \rightarrow \infty} a_m= \lim_{m \rightarrow \infty} \frac{m^2 + 1}{ m} = \infty.
$$
\end{example}
\medskip

As we have remarked, the definition of the limit of a sequence is included in the definition of the limit of a function. For emphasis, however, we shall give it in this special case. Let s be an infinite sequence of real numbers. Then \textbf{the limit as $n$ increases without bound of $s_n$ is equal to} $b$, written
$$
\lim_{n \rightarrow \infty}s_n = b,
$$
\noindent if, for $\varepsilon > 0$, there exists an integer $m$ in the domain of $s$ such that whenever $n > m$, then $|s_{n} - b| < \varepsilon$. The definition can be phrased geometrically as follows: The limit of $\{ s_n \}$ is $b$ if, given an arbitrary open interval $(b - \varepsilon, b + \varepsilon)$, all the numbers $s_{n}$ from some integer on, lie in that interval. Thus for the oscillating sequence 1, 0, 1, 0, 1, 0,... defined in (1), no limit exists.

The second topic in the section is the study of a convenient notation for the sum of a finite number of consecutive terms of a sequence. Let $a$ be a sequence (finite or infinite) of real numbers. If $m$ and $n$ are in the domain of the sequence, and if $m \leq n$, then the sum $a_m + a_{m+1}, + ... + a_n$ is called a series and is abbreviated $\sum_{i = m}^{n} a_{i}$. Thus 

$$
\sum_{i = m}^n a_i = a_m + a_{m+1} + ... + a_n.
$$
\noindent We call $\sum_{i = m}^{n}  a_i$ the \textbf{summation of $\{ ai \}$ from $m$ to $n$.}

%EXAMPLE 2. 
\begin{example} Let $\{ a_i \}$ be the sequence defined by $a_i = i^2$, for every positive integer $i$. Then  

$$
 \sum_{i = 1}^{5} a_i =  \sum_{i = 1}^{5} i^2 = 1^2 + 2^2 + 3^2 + 4^2 + 5^2 = 55.
$$
\noindent Another series defined from the same sequence is
$$
 \sum_{i = 3}^{6}a_i = \sum_{i = 3}^{6} i^2 = 3^2 + 4^2 + 5^2 + 6^2 = 86. 
$$
\noindent On the other hand, we might be interested in the sum of the squares of the first $n$ integers for an arbitrary positive integer $n$. This would be the series 

$$
 \sum_{i = 1}^{n} a_i =  \sum_{i = 1}^{n} i^2 = 1^2 + 2^2 + 3^2 + ... + n^2.
$$
\end{example}
\medskip

The symbol $i$ which appears in the series $ \sum_{i = m}^{n} a_i$ is called the \textbf{summation index}. It, too, is a dummy symbol, since the value of the series does not
%178 INTEGRATION [CHAP. 4
depend on $i$. Like the definite integral, $ \sum_{i = m}^{n} a_i$ depends on three things: the sequence $a$ (the function) and the two integers $m$ and $n$. Thus

$$
 \sum_{i = m}^{n} a_i =  \sum_{j = m}^{n} a_j =  \sum_{k = m}^{n} a_k = a_m + a_{m+1} + ... + a_n.
$$

%EXAMPLE 3. 
\begin{example}
Using the summation notation, write a series for the sum of all the odd integers
from 11 to 101. An arbitrary odd integer can be written in the form $2i + 1$ for some integer $i$. It is not hard to see, therefore, that one answer to the problem is given by the series

$$
 \sum_{i = 5}^{50} (2i + 1).
$$
\noindent Another is the series  

$$
 \sum_{i =6}^{51} (2i - 1).
$$
\end{example}
\medskip

It should be emphasized that the summation notation offers no new mathematical theory. It is merely a convenient shorthand for writing sums and manipulating them. The ability to manipulate comes from practice, but the techniques are based on the following properties:

\begin{theorem} %(2.1)
$$
\sum_{i = m}^{n} (a_{i} + b_{i}) = \sum_{i = m}^{n} a_{i} + \sum_{i = m}^{n} b_{i}.
$$
\end{theorem}

\begin{theorem} %(2.2) 
$$
\sum_{i = m}^{n} ca_{i} = c \sum_{i = m}^{n} a_{i}.
$$ 
\end{theorem}

\begin{theorem} %(2.3)
$$
\sum_{i = m}^{n} c = c (n - m + 1).
$$
\end{theorem}

\begin{proof}
The proofs are very simple. For (2.1) we have 
\begin{eqnarray*}
\sum_{i = m}^{n}(a_{i} + b_{i}) &=& (a_{m} + b_{m}) + (a_{m+1} + b_{m+1}) + ... + (a_{n} + b_{n})\\
&=& (a_m + a_{m +1} + ... + a_{n}) + (b_{m} + b_{m +1} + ... + b_{n})\\
&=& \sum_{i = m}^{n} a_{i} + \sum_{i = m}^{n} b_i.
\end{eqnarray*}
For (2.2),  
\begin{eqnarray*}
\sum_{i = m}^{n} ca_{i} &=& ca_{m} + ca_{m + 1} + ... + ca_{n}\\
                          &=& c(a_{m} + a_{m + 1} + ... + a_{n})\\
                          &=& c \sum_{i = m}^{n} a_{i}.
\end{eqnarray*}
To prove (2.3), one must understand that $\sum_{t = m}^{n} c$ means 
$\sum_{t = m}^{n} a_{i}$, where $\{ a_{i} \}$ is the constant sequence defined by $a_{i} = c$. Hence
\begin{eqnarray*}
\sum_{i = m}^{n} c = \sum_{i = m}^{n} a_{i }
&=& \overbrace{a_{m} + a_{m +1} + ... + a_{n}}^{n - m + 1 \;\mbox{terms}}\\
&=& c + c + ... + c \\
&=& c(n - m + 1).
\end{eqnarray*}
This completes the proof.
\end{proof}

There are two other summation identities which are useful and which we shall include. They are the formulas for the sum of the first $n$ positive integers and for the sum of the squares of the first $n$ positive integers:
\begin{theorem} %(2.4)
$$
\sum_{i = 1}^{n} i = \frac{n(n+1)} {2}.
$$
\end{theorem}

\begin{theorem} %(2.5)
$$
\sum_{i = 1}^{n} i^2 = \frac{n(n+1)(2n + 1)}{6}.
$$
\end{theorem}

\begin{proof}
There is a very clever proof of (2.4), which the great mathematician Carl Friedrich Gauss (1777-1855) is said to have figured out for himself in a few seconds in his first arithmetic class at the age of 10. Write the sum $S =\sum_{i = 1}^{n} i$, once in natural order and, underneath it, the sum in reverse order as follows:  
\begin{eqnarray*}
S &=& 1 + 2 + ... + (n - 1) + n,\\
S &=& n + (n + 1) + ... + 2 + 1.
\end{eqnarray*}
If each number on the right side of the first equation is added to the number directly beneath it, the sum is $n + 1$. Hence the sum of the two right sides is a series consisting of $n$ terms each equal to $n + 1$. It follows that
$$
2S = n(n + 1),
$$
from which (2.4) is an immediate corollary.

Formula (2.5) is probably most easily proved by induction on $n$. The proof is straightforward, and we omit it.
\end{proof}

%EXAMPLE 4. 
\begin{example} Evaluate  
 
\begin{description}
\item[(a) $\sum_{i = 1}^{n} (3i^2 + 5i - 2),$]
\item[(b) $\sum_{i = 1}^{n} \frac{(3i^2 + 5i -2)}{n^3}.$]
\end{description}
%180 INTEGRATION [CHAP. 4
\noindent Using the various properties of summation, we obtain for (a),  

\begin{eqnarray*}
\sum_{i = 1}^n(3i^2 + 5i - 2) &=& 3 \sum_{i = 1}^{n} i^2 + 5\sum_{i = 1}^{n} i - 2 \sum_{i = 1}^{n} 1\\ 
&=& 3 \frac{ n(n + 1)(2n + 1)}{6} + 5 \frac{n(n + 1)}{2} - 2n \\
&=& \frac {2n^3 + 3n^2 + n}{2} + \frac{ 5n^2 + 5n}{2} - \frac{ 4n}{2}  \\
&=& \frac{2n^3 + 8n^2 + 2n}{2} = n^3 + 4n^2 + n.
\end{eqnarray*}
\noindent Part (b) is really a trivial mod)fication of (a). The number $n^3$ which appears in the denominator is the same for each term in the sum, i.e., it is a constant, and can be factored out immediately. Thus

$$
\sum_{i = 1}^n \frac{3i^2 + 5i - 2}{n^3} = \frac{1}{n^3} \sum_{i = 1}^n (3i^2 + 5i - 2).
$$
\noindent Hence, using the answer from (a), we get 

\begin{eqnarray*}
\sum_{i = 1}^n \frac{ 3i^2 + 5i - 2}{n^3} &=& \frac{1}{n^3} (n^3 + 4n^2 + n)\\
                                            &=& 1 + \frac{4}{n} + \frac{1}{n^2}.
\end{eqnarray*}
\end{example}

We conclude the section with an example which combines the summation convention with the limit of an infinite sequence,
\medskip
%EXAMPLE 5.
\begin{example} For every positive integer $n$, let $S_n$ be defined by

$$
S_n = \sum_{i = 1}^n \frac{i^2 + 2}{n^3}.
$$
\noindent The numbers $S_1, S_2, S_3, ,...$ form an infinite sequence, and the problem is to evaluate $\lim_{n \rightarrow \infty} S_n$.  Using the properties of summation, we obtain

\begin{eqnarray*}
S_n &=& \frac{1}{n^3} \sum_{i = 1}^{n}  (i ^2 + 2) \\
       &=& \frac{1}{n^3} \Bigl( \sum_{i = 1} ^{n} i^2 + \sum_{i=1} ^{n} 2 \Bigr) \\
       &=& \frac{1}{n^3} \Bigl[\frac{n(n + 1)(2n + 1)}{6}  + 2n \Bigr] \\
       &=& \frac{1}{n^3} \Bigl( \frac{2n^3 + 3n^2 + n}{6} + \frac{12n}{6} \Bigr) \\
       &=& \frac{2n^3 + 3n^2 + 13n}{6 n^3}.
\end{eqnarray*}
 
%SEC. 2] SEQUENCES AND SUMMATIONS  181
\noindent Hence 
\begin{eqnarray*}
\lim_{n \rightarrow \infty} S_n &=& \lim_{n \rightarrow \infty} \frac{2n^3 + 3n^2 + 13n}{6n^3}\\
&=& \lim_{n \rightarrow \infty} \Bigl( \frac{1}{3} + \frac{1}{2n} + \frac{13}{6n^2} \Bigr) = \frac{1}{3},
\end{eqnarray*}
\noindent which is the answer to the problem. Frequently the notations are compounded; i.e., we write
$$
\lim_{n \rightarrow \infty} \sum_{i = 1}^{n} \frac{i^2 + 2}{n^3} = \frac{1}{3}.
$$
\end{example}
