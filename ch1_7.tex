\section{Derivatives of Polynomials and Rational Functions.}\label{sec 1.7} 
Computing $f'(x)$ from the definition of the derivative
by evaluating 
\[
\lim_{t \goesto 0} \frac{f(x + t) - f (x)}{t}
\]
can be quite a job.
In this section we shall develop a set of theorems
from which the derivatives of many functions,
including all polynomials and rational functions,
can be found easily and, what is more important,
in a completely routine way.

\begin{prop}\label{thm 1.7.1}
If $f$ and $g$ are differentiable functions,
then their sum $f + g$ is differentiable.
Moreover, $(f + g)' = f' + g'$.
\end{prop}

\begin{proof}
Let $a$ be a number in the domain of $f + g$.
Recall that by the definition of the sum of two functions

\begin{eqnarray*}
     (f + g)(a) &=& f(a) + g(a),   \\
(f + g)(a + t) &=& f(a + t) + g(a + t).
\end{eqnarray*}
Hence, by the definition of the derivative,

\begin{eqnarray*} 
(f + g)'(a)
&=&
\lim_{t \goesto 0} \frac{(f + g)(a + t) - (f + g)(a)}{t} \\
&=&
\lim_{t \goesto 0} \frac{f(a + t) + g(a + t) - (f(a) + g(a))}{t}\\
&=&
\lim_{t \goesto 0}
\Bigl( \frac{f(a + t) - f(a)}{t} + \frac{g(a + t) - g(a)}{t} \Bigr).
\end{eqnarray*}
It follows from the existence of $f'(a)$ and $g'(a)$
and the fact that the limit of a sum is the sum of the limits
[see the basic limit theorem \thref{1.4.1}(i)]
that we may continue the above sequence of equalities,
writing
\begin{eqnarray*}
&=& \lim_{t \goesto 0} {\frac {f(a + t) - f(a)}{t} } + \lim_{t \goesto 0} { \frac{g(a + t) - g(a)}{t} }  \\ 
&=& f'(a) + g'(a)  \\
&=& (f' + g')(a).
\end{eqnarray*}
This completes the proof.
\end{proof}

\begin{prop}\label{thm 1.7.2}
If $f$ is a differentiable function
and $c$ is a constant,
then $cf$ is differentiable and $(cf)' = cf'$.
\end{prop}

\begin{proof}
For any number a in the domain of $f$,
we have $(cf)(a) = cf(a)$.
Hence 
\begin{eqnarray*}
(cf)'(a)
&=&
\lim_{t \goesto 0} {\frac{ (cf)(a + t) - (cf)(a) }{t} }  \\
&=&
\lim_{t \goesto 0} {\frac{ cf(a + t) - cf(a)}{t} }   \\
&=&
\lim_{t \goesto 0} \Bigl(c \cdot {\frac{f(a + t) - f(a)}{t}} \Bigr).  
\end{eqnarray*}
By the basic limit theorem \thref{1.4.1}(ii)
and the assumed existence of $f'(a)$,
we can continue the chain of equalities,
writing
\begin{eqnarray*}
&=&
c \lim_{t \goesto 0} {\frac {f(a + t) - f(a)}{t} }  \\
&=&
cf'(a)\\
&=&
(cf')(a).
\end{eqnarray*}
This completes the proof.
\end{proof}

By taking $c = -1$, we get as a corollary of
\thref{1.7.1} and \thref{1.7.2}
that 
$$
(f - g)' = f'- g'. 
$$

\begin{prop}\label{thm 1.7.3}
The derivative of any constant function is the constant function zero; i.e., 
$$
c' = 0.
$$
\end{prop}

\begin{proof}
Recall that we allow ourselves the liberty of
denoting a real number and the constant function whose value
is that real number by the same letter.
Doing so here, we have 
\[
c(a) = c(a + t) = c, 
\]
for any numbers $a$ and $t$.
Hence 
\[
c'(a)
=
\lim_{t \goesto 0}
{\frac{c(a + t) - c(a)}{t} }
=
\lim_{t \goesto 0}{\frac{0}{t}} = 0.
\]
\end{proof}

\begin{example}
\label{exam 1.7.1}
Let $f(x) = x^3$, and $g(x) = \sqrt{x + 1} \; (x \geq -1)$,
and $h(x) = x^2 + 3$,
and suppose we are given the information that 
\begin{eqnarray*}
 f'(x) &=& 3x^2,    \\
 g'(x) &=& \frac{1}{2{\sqrt{x + 1}}},  \;\;\;  x > -1,\\
 h'(x) &=& 2x.
\end{eqnarray*}
It follows from the three theorems developed so far in this section
that the derivatives of the functions
\begin{quote}
(a)$5x^3 - 2 \sqrt{x+1}$,

(b)$x^2$,

(c)$3x^3 + 13x^2 + 7$,
\end{quote}
are, respectively, 
\begin{quote}
(a') $15x^2 - \frac{1}{\sqrt{x + 1}}$,

(b') $2x$, 

(c') $9x^2 + 26x$.
\end{quote} 
For example, to get (b'),
we write $x^2$ in the form $(x^2 + 3) - 3$.
Then
\[
(x^2)' = (x^2 + 3)' - 3' = 2x - 0 = 2x.
\]
The others are equally routine.
\end{example}

The next theorem
deals with the derivative of the product of two functions
and its conclusion is perhaps unexpected.
Note that it does not turn out that the derivative of a product
is the product of the derivatives.

\begin{prop}\label{thm 1.7.4}
If $f$ and $g$ are differentiable functions,
then their product $fg$ is differentiable.
Moreover, $(fg)' = f'g + g'f$.
\end{prop}

\begin{proof}
Let $a$ be a number in the domain of $fg$.
By the definition of the product of two functions we have
\[
(fg)(a) = f(a)g(a),
\]
\[
(fg)(a + t) = f(a + t)g(a + t).
\]
Hence
\begin{eqnarray*}
(fg)'(a)
&=&
\lim_{t \goesto 0} {\frac{(fg)(a + t) - (fg)(a)}{t} }  \\
&=&
\lim_{t \goesto 0} {\frac{f(a + t)g(a + t) - f(a)g(a)}{t} }.  
\end{eqnarray*}
The following algebraic manipulation
will enable us to put the above fraction
into a form in which we can see what the limit is:
\begin{eqnarray*} 
f(a + t)g(a + t) - f(a)g(a)
&=& f(a + t)g(a + t) - f(a)g(a + t) + f(a)g(a + t) - f(a)g(a) \\
&=& [f(a + t) - f(a)] g(a + t) + [g(a + t) - g(a)] f(a). 
\end{eqnarray*}
Thus
$$
(fg)'(a) =
\lim_{t \goesto 0}
\Bigl[{ \frac{f(a + t) - f(a)}{t} } g(a + t) +
{\frac{g(a + t) - g(a)}{t}} f(a) \Bigr].
$$
The limit of a sum of products is the sum of the products of the limits.
[Again, see the limit theorem \thref{1.4.1}.]
Moreover, $f'(a)$ and $g'(a)$ exist by hypothesis.
Finally, since $g$ is differentiable at $a$,
it is continuous there
[see Theorem \thref{1.6.1}];
and so $\lim_{t \goesto 0} {g(a + t) = g(a)}$.
We conclude that
\begin{eqnarray*}
(fg)'(a)
&=& \Bigl[ \lim_{t \goesto 0} {\frac{f(a + t) - f(a)}{t}} \Bigr] 
\lim_{t \goesto 0} {g(a + t)}   \\
& &  +
\Bigl[ \lim_{t \goesto 0}
{\frac{g(a + t) - g(a)}{t}} \Bigr] f(a)  \\
&=& f'(a)g(a) + g'(a)f(a) = (f'g + g'f)(a).
\end{eqnarray*}
This completes the proof of the product rule for differentiation.
\end{proof}

\begin{example}
\label{exam 1.7.2}
Suppose we are given the information that
the functions $f(x) = (x^2 + 2)^3$ and $g(x) = (x^2 + 2)^5$
have derivatives
\[
f'(x) = 6x(x^2 + 2)^2, 
\]
\[
g'(x) = 10x(x^2 + 2)^4.
\]
Find the derivative of $f(x)g(x) = (x^2 + 2)^8$.
Theorem \thref{1.7.4},
which is sometimes called Leibnitz's Rule,
states that
$$
(f(x)g(x))' = f'(x)g(x) + g'(x)f(x).
$$
Hence
\begin{eqnarray*} 
((x^2 + 2)^8)'
&=&
6x(x^2 + 2)^2(x^2 + 2)^5 + l0x(x^2 + 2)^4(x^2 + 2)^3  \\
&=& 16x(x^2 + 2)^7.
\end{eqnarray*}
\end{example}

The graph of the identity function $x$
is the straight line defined by the equation $y = x$,
which passes through the origin and has constant slope 1.
It follows that the derivative of the identity function
is the constant function 1. Thus
\begin{equation}
x'= 1. 
\label{eq1.7.1}
\end{equation}
We can apply the product (Leibnitz's) rule and obtain
$$
(x^2)' = (xx)' = x'x + x'x = 1x + 1x = 2x.
$$
Since $x^3 = x{x^2}$,
and we have just found the derivative of each factor,
we can use the product rule again to get
\begin{eqnarray*} 
(x^3)'
&=&
(x{x^2})' \\
&=&
x'{x^2} + (x^2)'x = 1{x^2} + 2x \cdot x \\
&=&
3x^2. 
\end{eqnarray*}
Again,
\begin{eqnarray*} 
(x^4)'
&=&
(xx^3)' \\
&=&
x'{x^3} + (x^3)'x = 1{x^3} + 3{x^2} \cdot x    \\
&=& 4x^3.
\end{eqnarray*}
These results suggest not only the statement of the next theorem,
but also how to prove it.

\begin{prop}\label{thm 1.7.5}
If $x$ is the identity function and $n$ is a positive integer,
then $(x^n)' = nx^{n-1}$.
\end{prop}

\begin{proof}
We have already proved the theorem for $n = 1$.
(Actually we have also proved it for $n$ = 2, 3, and 4,
but for the moment this is irrelevant.)
Suppose we had proved it for all positive integers up to and including $k$.
In particular, we would know that $(x^k)' = kx^{k-1}$.
We could then use the product rule to derive
$(x^{k+1})'
= (xx^k)'
= x'x^k + (x^k)'x
= 1x^k + kx^{k-1} \cdot x
= (k + 1)x^k$. 
Thus the theorem is true for $n$ = 1,
and if it is true for $n = k$,
it is then also true for $n = k + 1$.
We conclude that the theorem holds for every positive integer $n$.
\end{proof}

This is an example of a proof by mathematical induction.
The reasoning can be paraphrased like this:
Suppose I know that I can get on the bottom rung of a ladder.
Suppose further that, if I am standing on any rung,
then I can reach the next rung.
It follows that I can climb the ladder.

%EXAMPLE 3. 
\begin{example}
Find the derivatives of the polynomials:
\begin{eqnarray*}
f(x) &=& x^3- 2,                \\
g(x) &=& 3x^2 + 7x - 13,  \\
y &=& 4x^4 + 3x^3 + 2x^2 + x,\\
s &=& \frac{1}{2}gt^2
\provx{($g$ is a constant, and $t$ is an independent variable)} . 
\end{eqnarray*} 

The answers are immediate:
\begin{eqnarray*}
 f'(x) &=& 3x^2,                 \\
g'(x) &=& 6x + 7,               \\
    y' &=& 16x^3 + 9x^2 + 4x + 1, \\
    s' &=& gt.
\end{eqnarray*} 

It should be clear that, as a result of the rules developed so far,
the derivative of any polynomial function can be computed immediately
and in a purely mechanical way.
We turn next to the derivative of a ratio.
\end{example}

\begin{prop}\label{thm 1.7.6}
If $f$ and $g$ are differerentiable functions, then the quotient $\frac{f}{g}$ is differentiable [if $g(a) = 0$, then $\Bigl( \frac{f}{g} \Bigr) (a)$ is not defined]. Moreover, 

$$
\Bigl( \frac{f}{g} \Bigr)' = \frac{gf'- fg'}{g^2} .
$$
\end{prop}

\begin{proof}
We first prove that the function $\frac{1}{g}$ is differentiable at a number $a$ in its domain provided $g(a) \neq 0$. By definition,
\begin{eqnarray*}
   \Bigl( \frac{1}{g} \Bigr)'(a) 
&=& \lim_{t \goesto 0} {\frac{\Bigl( \frac{1}{g} \Bigr) (a + t) - \Bigl( \frac{1}{g} \Bigr)(a)}{t} }  \\
&=& \lim_{t \goesto 0} {\frac{\frac{1}{g(a + t)} - \frac{1}{g(a)} }{t} }.
\end{eqnarray*} 
Note that since $g$ is continuous at $a$ [see Theorem (6.1)] and $g(a) \neq 0$, we know that $g(a + t) \neq 0$ for sufficiently small values of $t$. Since
\begin{eqnarray*}
\frac{\frac{1}{g(a + t)} - \frac{1}{g(a)} }{t} 
&=& \frac{(g(a) - g(a + t)}{t g(a) g(a+1) }\\
&=& -\Bigl( \frac{1}{g(a)g(a+t)} \Bigr) \Bigl(\frac{g(a + t) - g(a)}{t} \Bigr),
\end{eqnarray*}
we have
\[
\Bigl(\frac{1}{g} \Bigr)'(a) = \lim_{t \goesto 0} {\Bigl[ - \Bigl(\frac{1}{g(a)g(a + t)} \Bigr) \Bigl(\frac{g(a + t) - g(a)}{t} \Bigr) \Bigr]}.
\]
The derivative $g'(a)$ exists by hypothesis, and $\lim_{t \goesto 0} g(a + t) = g(a) \neq 0$. The basic limit theorem (4.1) therefore
implies that
\begin{eqnarray*}
\Bigl(\frac{1}{g} \Bigr)'(a) 
&=& - \Bigl( \frac{1}{g(a) \lim_{t \goesto 0} g(a + t)} \Bigr) \Bigl(\lim_{t \goesto 0} {\frac{g(a + t) - g(a)}{t} \Bigr) }   \\
&=& -\frac{1}{(g(a))^2} g'(a) = -\frac{g'(a)}{(g(a))^2}.
\end{eqnarray*}
This proves the differentiability of the function $-\frac{1}{g}$ and also establishes the following special case of the quotient rule:  
\begin{equation}
\Bigl( \frac{1}{g} \Bigr)' = -\frac{g'}{g^2}.
\label{eq1.7.2}
\end{equation}
The general form of (7.6) can now be obtained using the product rule: 
\begin{eqnarray*}
\Bigl( \frac{f}{g} \Bigr)' 
&=& \Bigl(f \cdot \frac{1}{g} \Bigr)' = f' \cdot {\frac{1}{g}} + f \cdot
\Bigl( \frac{1}{g} \Bigr)' \\
&=&  \frac{f'}{g} - f \cdot \frac{g'}{g^2} = \frac{gf' - fg'}{g^2}.
\end{eqnarray*}
This completes the proof.
\end{proof}

%EXAMPLE 4. 
\begin{example}
Find the derivatives of the following rational functions:

\begin{eqnarray*} 
f(x) &=& \frac{x^2 + 1}{x},           \\
g(y) &=& \frac{y^2 - 3y + 1}{y^3 - 1},\\
h(s) &=& \frac{1}{s^3}                \\
   w &=& \frac{u - a}{u - b} \;\;\; \mbox{($a$ and $b$ are constants and $u$ is an independent variable)}.
\end{eqnarray*}

Applying our six rules, we get

\begin{eqnarray*} 
 f'(x) &=& \frac{x \cdot 2x - (x^2 + 1) \cdot 1}{x^2} = \frac{x^2 -1}{x^2},  \\
 g'(y) &=& \frac{(y^3 - 1)(2y - 3) - (y^2 - 3y + 1)3y^2}{(y^3 - 1)^2}
          = \frac{-y^4 + 6y^3 - 3y^2 -2y + 3}{(y^3 - 1)^2} ,                   \\
 h'(s) &=& \frac{-3s^2}{s^6} = - \frac {3}{s^4} ,                              \\
    w' &=& \frac{(u - b) \cdot 1 - (u - a) \cdot 1}{(u - b)^2} = \frac{a - b}{(u - b)^2}.
\end{eqnarray*}
\end{example}

It is important to realize that the symmetry present in the product rule is missing in the quotient rule. For the former, order is immaterial: The prime appears once on one factor and once on the other, and that is all there is to remember. This is not so for the quotient rule, however, where the wrong order will result in the wrong sign in the answer. There is no help for it but to memorize the formula precisely.

The formula for the derivative of $x^n$ has been proved only if $n$ is a non-negative integer. (It holds for $n = 0$ because $x^0 = 1$.) The next theorem enlarges the scope of the formula to include all integers.

\begin{prop}\label{thm 1.7.7}
If $x$ is the identity function and $n$ is an integer
(positive, negative, or zero),
then $(x^n)' = nx^{n-1}$. 
\end{prop}

\begin{proof}
We shall assume that $n$ is a negative integer, since the theorem is known  to be true otherwise. Then $m = -n$ is a positive integer, and $x^n =\frac{1}{x^m}$. Using (2) and (7.5), we get 
\begin{eqnarray*}
   (x^n)' &=& \Bigl( \frac{1}{x^m} \Bigr)' = -\frac{(x^m)'}{x^{2m}} = -\frac{mx^{m-1}}{x^{2m}}  \\
          &=& (-m)x^{-m-1} = nx^{n-1}. 
\end{eqnarray*}
This completes the proof.
\end{proof}

Thus, for example, if $f(x) = x^{-7}$, then $f'(x) = - 7x^{-8}$. In Section \secref{1.8} we shall show that the formula is actually valid, not only for integers, but for any rational number $n$. Finally, in Chapter \chref{5} we shall prove that $(x^a)' = ax^{a-1}$, for any real number $a$.

Let us summarize in a single list the theorems that we have developed for finding derivatives. To provide practice, we shall this time employ the alternative $\frac{d}{dx}$ notation.  Let $u$ and $v$ be differentiable functions of $x$, and $ c$ a constant. Then

 
\begin{prop}
\label{thm ddxrules}

\begin{quote}
\begin{description}
\item[(i)] $\frac{d(u + v)}{dx} = \frac{du}{dx} + \frac{dv}{dx},$

\item[(ii)] $\frac{d(cu)}{dx} = c \frac{du}{dx},$

\item[(iii)] $\frac{dc}{dx}  = 0,$

\item[(iv)] $\frac {d(uv)}{dx}  = u \frac{dv}{dx} + v \frac{du}{dx},$

\item[(v)] $\frac{dx^n}{dx} = nx^{n-1}, \provx{where $n$ is any integer,}$ 


\item[(vi)] $\frac{d}{dx} \Bigl( \frac{u}{v} \Bigr) = \frac{v {\frac{du}{dx}} - u {\frac{dv}{dx}} }{v^2}.$
\end{description}
\end{quote}  
\end{prop}

Note that we have proved these theorems for arbitrary differentiable functions $u$ and $v$, not just for polynomials and rational functions.
%EXAMPLE 5. 
\begin{example} Let

\begin{eqnarray*}
y &=& 2x^3+ 7x+ 1,          \\
u &=& x^7+ \frac{1}{x^5},   \\
s &=& \frac{3t^2 + 2t + 1}{t-4}.
\end{eqnarray*}
Then 

\begin{eqnarray*}
 \frac{dy}{dx} &=& 6x^2 + 7,            \\
 \frac{du}{dx} &=& 7x^6 - \frac{5}{x^6},\\
\frac{ds}{dt}  &=& \frac{ (t-4)(6t + 2) - (3t^2 + 2t + 1) \cdot 1}{(t-4)^2}
=\frac{3t^2 - 24t - 9}{(t-4)^2} .
\end{eqnarray*}
\end{example}

We have seen in this section that the derivative of a polynomial is another polynomial, and the derivative of a rational function is a new rational function. Once we have found the derivative $f'$ of any function $f$, we can go on and find the derivative of $f'$. The new function, denoted $f''$, is called the \dt{second derivative} of $f$. Clearly,

$$
f''(a) = \lim_{t \goesto 0} {\frac{f' (a + t) - f'(a)}{t}} .
$$ 
The \dt{third derivative,} written $f'''$, is the derivative of the second derivative, and, in principle, we can go on forever and form derivatives of as high order as we like. It would obviously be absurd to write the seventeenth derivative with seventeen primes, so we adopt the alternative rotation $f^{(n)}$ for the nth derivative of $f$.

The differential notation for the higher derivatives is based on the idea that  $\frac{d}{dx}$ is a function, sometimes called an operator, which assigns to a function its derivative with respect to $x$. Hence we write

\begin{eqnarray*}
\frac{d}{dx} \Bigl(\frac{df}{dx} \Bigr) &=& \frac{d^{2}y}{dx^2} = f'',\\
\frac{d}{dx} \Bigl( \frac{d^{2}f}{dx^2} \Bigr) &=& \frac{d^{3}y}{dx^3} = f''',\\
                              \frac{d^{n}f}{dx^n} &=& f^{(n)},\\
                           \frac{d^{2}f}{dx^2}(a) &=& f''(a), \; \mbox{etc.}
\end{eqnarray*}

In addition, if a variable is used to denote a function, for example, if $y = f(x)$, we also use the expressions
$$
\frac{{d^2}y}{dx^2} = y'' = f'',  \;\;\;  \frac{{d^n}y}{dx^n}= y^(n) = f^(n), 
\;\;\; \mbox{etc.}
$$

%EXAMPLE 6. 
\begin{example} Let $f(x) = x^3 + 3x^2 + 1$. Then 

\begin{eqnarray*}
      f'(x) &=& 3x^2 + 6x, \\
      f''(x) &=& 6x + 6, \\
     f'''(x) &=& 6,\\
f^{(n)}(x)&=& 0, \;\;\; \mbox{if}\;\;\;   n>3.
\end{eqnarray*}
As another example, let $y = \frac{1}{x+ 1}$. Then

\begin{eqnarray*}
       \frac{dy}{dx} &=& -\frac{1}{(x + 1)^2},\\
\frac{d^2y}{dx^2} &=& \frac{1}{(x+1)^3}, \\
\vdots\\
\frac{dy^n}{dx^n} &=& \frac{(-1)^n n!}{(x+1)^{n+1}},  \;\;\; n!= 1 \cdot 2 \cdot 3 \cdots
n.
\end{eqnarray*}
\end{example}
